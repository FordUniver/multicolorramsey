\documentclass[12pt,reqno]{amsart}
\usepackage{amsmath,amssymb,amsthm,bbm,mathtools,calc,verbatim,enumitem,tikz,url,mathrsfs,cite,fullpage,hyperref}
\usepackage{float}
\usepackage{subcaption}
%\usepackage{setspace}
\renewcommand{\baselinestretch}{1.1}
\addtolength{\footskip}{\baselineskip/2}

%\usepackage{showlabels}
\usepackage{comment}

%\parskip=12pt plus 1pt

\newcommand\marginal[1]{\marginpar{\raggedright\parindent=0pt\tiny #1}}

\newtheorem{theorem}{Theorem}[section]
\newtheorem{lemma}[theorem]{Lemma}
\newtheorem{corollary}[theorem]{Corollary}
\newtheorem{prop}[theorem]{Proposition}
\newtheorem{observation}[theorem]{Observation}
\newtheorem{construction}[theorem]{Construction}

\newtheorem{conjecture}[theorem]{Conjecture}
\newtheorem{question}[theorem]{Question}
\newtheorem{obs}[theorem]{Observation}
\newtheorem{claim}[theorem]{Claim}
\newtheorem{fact}[theorem]{Fact}

\theoremstyle{definition}
\newtheorem{defn}[theorem]{Definition}
\newtheorem*{RBalg}{The Multicolour Book Algorithm}
\theoremstyle{remark}
\newtheorem{remark}[theorem]{Remark}

\newenvironment{clmproof}[1]{\begin{proof}[Proof of Claim~\ref{#1}]\let\qednow\qedsymbol\renewcommand{\qedsymbol}{}}{\; \qednow \end{proof}}

\newcommand\N{\mathbb{N}}
\newcommand\R{\mathbb{R}}
\newcommand\Z{\mathbb{Z}}
\newcommand\cA{\mathcal{A}}
\newcommand\cB{\mathcal{B}}
\newcommand\cN{\mathcal{N}}
\newcommand\cP{\mathcal{P}}
\newcommand\cQ{\mathcal{Q}}
\newcommand\cZ{\mathcal{Z}}
\newcommand\rN{\tilde{N}}
\newcommand\cT{\mathcal{T}}
\newcommand\cE{\mathcal{E}}
\def\Pr{\mathbb{P}}
\def\cS{\mathcal{S}}
\newcommand\Ex{\mathbb{E}}
\newcommand\id{\hbox{$1\mkern-6.5mu1$}}
\newcommand\lcm{\operatorname{lcm}}
\newcommand\eps{\varepsilon}

\renewcommand{\leq}{\leqslant}
\renewcommand{\geq}{\geqslant}
\renewcommand{\le}{\leqslant}
\renewcommand{\ge}{\geqslant}
\renewcommand{\to}{\rightarrow}
\renewcommand{\Re}{\re}

\def\ds{\displaystyle}

\def\eps{\varepsilon}
	\def\p{\partial}
		
	\def\HH{\mathcal{H}}
	\def\E{\mathbb{E}}
	\def\C{\mathbb{C}}
		\def\cM{\mathcal{M}}
		\def\cF{\mathcal{F}}
		\def\cI{\mathcal{I}}
	\def\R{\mathbb{R}}
	\def\bS{\mathbb{S}}
	\def\bH{\mathbb{H}}	
	\def\Z{\mathbb{Z}}
	\def\N{\mathbb{N}}
	\def\PP{\mathbb{P}}
	\def\1{\mathbbm{1}}
	\def\l{}
	\def\k{\kappa}
	\def\w{\omega}
	\def\s{\sigma}
	\def\t{\theta}
	\def\a{\alpha}
	\def\g{\gamma}
	\def\z{\zeta}	
	\def\zbar{\bar{z}}
	\def\<{\langle}
	\def\>{\rangle}	
	%\def\endproof{{\hfill $\square$} }
	\def\Xt{\widetilde{X}}
	\def\Pt{\widetilde{P}}
	
	\def\cN{\mathcal{N}}
	\def\cC{\mathcal{C}}
	\def\cD{\mathcal{D}}
	\def\cR{\mathcal{R}}
	\def\cB{\mathcal{B}}
	\def\cG{\mathcal{G}}
	\def\EE{\mathbb{E}}
		\def\FF{\mathbb{F}}
		\def\T{\mathbb{T}}
	\def\cA{\mathcal{A}}
	\def\cQ{\mathcal{Q}}	
	\def\cC{\mathcal{C}}
	\def\F{\mathbb{F}}
	\def\tm{\tilde{\mu}}
	\def\ts{\tilde{\sigma}}
	\def\Q{\mathcal{Q}}
	\def\vp{\varphi}

\pagestyle{plain}

\begin{document}

\title{Upper bounds for multicolour Ramsey numbers}

\author{Paul Balister \and B\'ela Bollob\'as \and Marcelo Campos \and Simon Griffiths \and Eoin Hurley\and Robert Morris \and Julian Sahasrabudhe \and Marius Tiba}

%\author{Paul Balister}
\address{Mathematical Institute, University of Oxford, Radcliffe Observatory Quarter, Woodstock Road, Oxford, OX2 6GG, UK}
\email{Paul.Balister@maths.ox.ac.uk}

%\author{B\'ela Bollob\'as}
\address{Department of Pure Mathematics and Mathematical Statistics, Wilberforce Road, Cambridge, CB3 0WA, UK, and Department of Mathematical Sciences, University of Memphis, Memphis, TN 38152, USA}
\email{bb12@cam.ac.uk}

%\author{Marcelo Campos}
\address{Trinity College, Cambridge, CB2 1TQ, UK}
\email{mc2482@cam.ac.uk}

%\author{Simon Griffiths}
\address{Departamento de Matem\'atica, PUC-Rio, Rua Marqu\^{e}s de S\~{a}o Vicente 225, G\'avea, 22451-900 Rio de Janeiro, Brasil}
\email{simon@mat.puc-rio.br}

%\author{Eoin Hurley}
\address{Mathematical Institute, University of Oxford, Radcliffe Observatory Quarter, Woodstock Road, Oxford, OX2 6GG, UK}
\email{hurley@maths.ox.ac.uk}

%\author{Robert Morris}
\address{IMPA, Estrada Dona Castorina 110, Jardim Bot\^anico, Rio de Janeiro, 22460-320, Brasil}\email{rob@impa.br}

%\author{Julian Sahasrabudhe}
\address{Department of Pure Mathematics and Mathematical Statistics, Wilberforce Road, Cambridge, CB3 0WA, UK}
\email{jdrs2@cam.ac.uk}

%\author{Marius Tiba}
\address{Department of Mathematics, King’s College London, Strand, London, WC2R 2LS, UK}
\email{marius.tiba@kcl.ac.uk}

\thanks{SG was partially supported by FAPERJ (Proc.~201.194/2022) and by CNPq (Proc.~407970/2023-1); EH by the Gravitation Programme NETWORKS (024.002.003) of NWO; and RM by FAPERJ (Proc.~E-26/200.977/2021) and by CNPq (Procs~303681/2020-9 and~407970/2023-1)}

\begin{abstract}
The $r$-colour Ramsey number $R_r(k)$ is the minimum $n \in \N$ such that every $r$-colouring of the edges of the complete graph $K_n$ on $n$ vertices contains a monochromatic copy of $K_k$. We prove, for each fixed $r \ge 2$, that 
$$R_r(k) \le e^{-\delta k} r^{rk}$$ 
for some constant $\delta = \delta(r) > 0$ and all sufficiently large $k \in \N$. For each $r \ge 3$, this is the first exponential improvement over the upper bound of~Erd\H{o}s and Szekeres from 1935. In the case $r = 2$, it gives a different (and significantly shorter) proof of a recent result of Campos, Griffiths, Morris and Sahasrabudhe.  
\end{abstract}

\maketitle