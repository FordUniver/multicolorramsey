\section{The Proof of the Key Lemma}\label{sec:key:lemma}

In this section we will prove Lemma~\ref{key:lemma}. The main step in the proof is the following geometric lemma, which is the most important new ingredient in the proof of Theorem~\ref{thm:Ramsey:multicolour}. The proof of the geometric lemma is (to us, at least) surprisingly short and simple. 

Recall that we fixed $n,r \in \N$, and defined $\beta = 3^{-4r}$ and $C = 4r^{3/2}$. In this section we will write $\langle\cdot,\cdot\rangle$ to denote the standard inner product on $\R^n$.  

\begin{lemma}\label{lem:lambda}
Let\/ $U$ and\/ $U'$ be i.i.d.~random variables taking values in a finite set~$X$, and let $\sigma_1,\ldots,\sigma_r \colon X \to \R^n$ be arbitrary functions. There exists $\lambda\ge-1$ and\/ $i\in[r]$ such that
$$\Pr\Big( \big\langle \sigma_i(U),\sigma_i(U') \big\rangle \ge \lambda \, \text{ and } \, \big\langle \sigma_j(U), \sigma_j(U') \big\rangle \ge -1 \, \text{ for all } \, j \ne i \Big) \ge \beta e^{- C\sqrt{\lambda + 1}}.$$
\end{lemma}

The idea of the proof is as follows. We will first observe (see Lemma~\ref{lem:moments}) that all of the moments of the inner products $\langle \sigma_i(U),\sigma_i(U') \rangle$ are positive. The key step is then to define a function $f \colon \R^r \to \R$ (see~\eqref{def:f}) whose Taylor expansion has non-negative coefficients, that is only positive close to the positive quadrant, and that does not grow too fast. With this function in hand, the lemma will then follow from a simple calculation. 

We begin with the following simple but key observation. 

\begin{lemma}\label{lem:moments}
Let $U$ and\/ $U'$ be i.i.d.~random variables taking values in a finite set~$X$, and let\/ $\sigma_1,\ldots,\sigma_r \colon X \to \R^n$ be arbitrary functions. Then
$$\Ex\Big[ \big\langle \sigma_1(U),\sigma_1(U') \big\rangle^{\ell_1} \cdots \big\langle \sigma_r(U),\sigma_r(U') \big\rangle^{\ell_r} \Big] \ge 0.$$
for every $(\ell_1,\dots,\ell_r) \in \Z^r$ with $\ell_1,\dots,\ell_r \ge 0$.
\end{lemma}

\begin{proof}
To simplify the notation, let us write 
$$\big\langle \sigma_1(U),\sigma_1(U') \big\rangle^{\ell_1} \cdots \big\langle \sigma_r(U), \sigma_r(U') \big\rangle^{\ell_r} = \prod_{i = 1}^\ell \big\langle \sigma_{a_i}(U), \sigma_{a_i}(U') \big\rangle $$
where $\ell = \ell_1 + \cdots + \ell_r$ and $(a_1,\dots,a_\ell)$ is such that $\big| \big\{ i \in [\ell] : a_i = j \big\} \big| = \ell_j$ for each $j \in [r]$. Now set
$$Z = \sigma_{a_1}(U) \otimes \sigma_{a_2}(U) \otimes \cdots \otimes \sigma_{a_\ell}(U) \quad \text{and} \quad Z' = \sigma_{a_1}(U') \otimes \sigma_{a_2}(U') \otimes \cdots \otimes \sigma_{a_\ell}(U'),$$ 
and note that 
$$\big\langle Z, Z' \big\rangle = \prod_{i = 1}^\ell \big\langle \sigma_{a_i}(U), \sigma_{a_i}(U') \big\rangle,$$ 
since $\langle u_1 \otimes \cdots \otimes u_r, v_1 \otimes \cdots \otimes v_r \rangle = \langle u_1, v_1 \rangle  \cdots \langle v_r , u_r \rangle $ for any vectors $u_i,v_i \in \R^n$. Finally, note that $Z$ and $Z'$ are independent and identically distributed random vectors, and therefore
$$\Ex \big[ \langle Z, Z' \rangle \big] = \Ex_{Z'} \big[ \Ex_Z \big[ \langle Z, Z' \rangle \big] \big] = 
\Ex_{Z'} \big[ \big\langle \Ex[Z], Z' \big\rangle \big] = \big\langle \Ex[Z], \Ex[Z'] \big\rangle \ge 0,$$ 
as required, where the final inequality holds because $\Ex[Z] = \Ex[Z']$. 
\end{proof}

Next, we will need the following special function: 
\begin{equation}\label{def:f}
f(x_1,\dots,x_r) = \sum_{j = 1}^r x_j \prod_{i \ne j} \big( 2 + \cosh\sqrt{x_i} \big),
\end{equation}
where we define $\cosh \sqrt{x}$ via its Taylor expansion
$$\cosh\sqrt{x} = \sum_{n = 0}^\infty \frac{x^n}{(2n)!}.$$ 
In particular, note that all of the coefficients of the Taylor expansion of $f$ are non-negative. The function $f$ also satisfies the following two inequalities. 

\begin{lemma}\label{lem:special:function}
Let $r \in \N$. The function $f \colon \R^r \to \R$ defined in~\eqref{def:f} satisfies
$$
f(x_1,\dots,x_r) \le \left\{\begin{array}{cl}
3^r r \exp\bigg( \displaystyle\sum_{i = 1}^r \sqrt{ x_i + 3r } \bigg) \quad & \text{if } \,\, x_i \ge - 3r \,\text{ for all }\, i \in [r];\\[+3ex]
-1 & \text{otherwise.} 
\end{array} \right.
$$
\end{lemma}

\begin{proof}
Observe first that $x \le 2 + \cosh\sqrt{x} \le 3e^{\sqrt{x}}$ for every $x \ge 0$, 
%since $2 + \cosh\sqrt{x} - x \ge 3 - \frac{x}{2} + \frac{x^2}{24} \ge 3(1 - \frac{x}{12})^2 \ge 0 
that 
$$f(x_1,\dots,x_r) = \bigg( \prod_{i = 1}^r \big( 2 + \cosh \sqrt{x_i} \big) \bigg) \sum_{j = 1}^r \frac{x_j}{2+\cosh\sqrt{x_j}},$$
and that $\cosh \sqrt{x} = \cos \sqrt{-x}$, and hence $-1 \le \cosh \sqrt{x} \le 1$, for all $x < 0$. It follows that
$$f(x_1,\dots,x_r)\le r \prod_{i=1}^r 3e^{\sqrt{x_i + 3r}} =3^r r \exp\bigg(\sum_{i=1}^r \sqrt{x_i + 3r} \bigg),$$
for every $x_1,\ldots,x_r \ge -3r$, as claimed. Moreover, if $x_i \le -3r$, then
$$\sum_{j = 1}^r \frac{x_j}{2+\cosh\sqrt{x_j}} \le \frac{x_i}{3} + r - 1 \le - 1.$$
Since $2 + \cosh \sqrt{x} \ge 1$ for every $x \in \R$, it follows that $f(x_1,\dots,x_r) \le -1$, as required. 
\end{proof}

\pagebreak

\begin{remark}
The key idea in the lemma above was to find an entire function, $2 + \cosh\sqrt{x}$, which $(a)$ has a Taylor expansion at $x = 0$ with non-negative coefficients; $(b)$ is bounded on the negative real axis; and $(c)$ does not grow too quickly on the positive axis. The Phragm\'en--Lindel\"of theorem %from complex analysis 
implies that functions satisfying $(a)$ and $(b)$ must grow at least as fast as $\exp\big( \Omega(\sqrt{x}) \big)$ on the positive real axis.  Thus the bound on the growth of $f$ given by the lemma is essentially best possible for constructions of this type.
\end{remark}

We are now ready to prove our geometric lemma. 

\begin{proof}[Proof of Lemma~\ref{lem:lambda}]
For each $i \in [r]$, define $Z_i = 3r\big\langle \sigma_i(U),\sigma_i(U') \big\rangle$, and observe that, by Lemma~\ref{lem:moments} and linearity of expectation, we have
\begin{equation}\label{eq:Ex:f:positive} 
\Ex\big[ f\big( Z_1,\ldots,Z_r \big) \big] \ge 0,
%\Ex\Big[ f\Big( 3r\big\langle \sigma_1(U),\sigma_1(U') \big\rangle, \dots, 3r\big\langle \sigma_r(U),\sigma_r(U') \big\rangle \Big) \Big] \ge 0,
\end{equation}
since all of the coefficients of the Taylor expansion of $f$ are non-negative. By Lemma~\ref{lem:special:function}, it follows that if we define $E$ to be the event that $Z_i \ge -3r$ %$\langle \sigma_i(U),\sigma_i(U') \rangle \ge - 1$ 
for every $i \in [r]$, then 
\begin{equation}\label{eq:eventE:inequality} 
3^r r \cdot \Ex\bigg[ \exp\bigg( \sum_{i = 1}^r \sqrt{ Z_i + 3r } \bigg) \mathbf{1}_E \bigg] \ge 1 - \Pr(E),
%3^r r \cdot \Ex\bigg[ \exp\bigg( \sqrt{3r} \sum_{i = 1}^r \sqrt{ \big\langle \sigma_i(U),\sigma_i(U') \big\rangle + 1 } \bigg) \id_E \bigg] \ge 1 - \Pr(E),
\end{equation}
where $\mathbf{1}_E \in \{0,1\}$ denotes the indicator of the event $E$, since the left-hand side bounds the %(positive) 
contribution to the expectation on the left-hand side of~\eqref{eq:Ex:f:positive} due to the event $E$, and the right-hand side bounds the %(negative) 
contribution to the expectation due to the event $E^c$. 

The result now follows from a simple calculation. First, note that 
$$\Pr(E) = \Pr\Big( Z_i \ge - 3r \, \text{ for all } \, i \in [r] \Big),$$
%$$\Pr(E) = \Pr\Big( \big\langle \sigma_j(U), \sigma_j(U') \big\rangle \ge -1 \, \text{ for all } \, j \in [r] \Big),$$
so if $\Pr(E) \ge \beta$, then the claimed inequality holds with $\lambda = -1$. We claim that if $\Pr(E) \le \beta$, then there exists $\lambda \ge -1$ such that
\begin{equation}\label{eq:max:big:and:E}
 \Pr\Big( \big\{ M \ge \lambda \big\} \cap E \Big) \ge \beta r e^{-C\sqrt{\lambda + 1}},
\end{equation}
where $M = \max \big\{ \big\langle \sigma_i(U),\sigma_i(U') \big\rangle : i \in [r] \big\}$, 
%$M = \max \big\{ Z_i / 3r : i \in [r] \big\}$, %
and therefore (by the union bound) there exists an $i \in [r]$ as required. Indeed, if no such $\lambda$ exists, then for any constant $c \le C - 1$, we have
\begin{align*}
\Ex\Big[ \exp\big( c \sqrt{M + 1} \big) \mathbf{1}_E \Big]
& \le \, \Pr(E) + \int_{-1}^\infty \Pr\Big( \big\{ M \ge \lambda \big\} \cap E \Big) \cdot \frac{c}{2\sqrt{\lambda + 1}} \cdot e^{c \sqrt{\lambda + 1}} \,\mathrm{d}\lambda\\
& \le \, \beta + \beta r \int_{-1}^\infty \frac{c}{2\sqrt{\lambda + 1}} \cdot e^{- \sqrt{\lambda + 1}} \,\mathrm{d}\lambda \, \le \, \beta (cr + 1),
\end{align*}
where in the first step we used Fubini's theorem, in the second we used $\Pr(E) \le \beta$ and $c \le C - 1$, and in the final step we used the fact that $\int_0^\infty \frac{1}{2\sqrt{x}} e^{-\sqrt{x}} \, \mathrm{d}x = 1$. In particular, applying this with $c = \sqrt{3}r^{3/2}$, and recalling that $C = 4r^{3/2} \ge c + 1$ and $\beta = 3^{-4r}$, it follows that the left-hand side of~\eqref{eq:eventE:inequality} is at most $3^r r \cdot 3^{-4r} \big( \sqrt{3}r^{5/2} + 1 \big) \le 1/3$, which contradicts our assumption that $\Pr(E) \le \beta$. Hence there exists $\lambda \ge -1$ such that~\eqref{eq:max:big:and:E} holds, as claimed.
\end{proof}

It is now straightforward to deduce Lemma~\ref{key:lemma} from Lemma~\ref{lem:lambda}. 

%\pagebreak

\begin{proof}[Proof of Lemma~\ref{key:lemma}] 
For each colour $i \in [r]$, define a function $\sigma_i \colon X \to \R^{Y_i}$ as follows: for each $x \in X$, choose a set $N'_i(x) \subset N_i(x) \cap Y_i$ of size exactly $p_i|Y_i|$, where $p_i = p_i(X,Y_i)$, and set
$$\sigma_i(x) = \frac{\id_{N'_i(x)} - p_i\id_{Y_i}}{\sqrt{\alpha_ip_i|Y_i|}},$$
where $\id_S \in \{0,1\}^{Y_i}$ denotes the indicator function of the set $S$. Note that, for any $x,y\in X$,
$$\big\langle \sigma_i(x),\sigma_i(y) \big\rangle \ge \lambda \quad \Leftrightarrow \quad |N'_i(x) \cap N'_i(y)|\ge \big( p_i + \lambda\alpha_i \big) p_i |Y_i|.$$
Now, by Lemma~\ref{lem:lambda}, there exists $\lambda \ge -1$ and colour $\ell \in [r]$ such that
$$\Pr\Big( \big\langle \sigma_\ell(U),\sigma_\ell(U') \big\rangle \ge \lambda \, \text{ and } \, \big\langle \sigma_i(U), \sigma_i(U') \big\rangle \ge -1 \, \text{ for all } \, i \ne \ell \Big) \ge \beta e^{- C\sqrt{\lambda + 1}}.$$
where $U$, $U'$ are independent random variables distributed uniformly in the set~$X$. Hence there exists a vertex $x \in X$ and a set $X' \subset X$ such that, 
$$|X'| \ge \beta e^{- C \sqrt{\lambda + 1}} |X| \qquad \text{and} \qquad |N'_\ell(x) \cap N'_\ell(y)| \ge \big( p_\ell + \lambda\alpha_\ell \big) p_\ell |Y_\ell |$$
for every $y \in X'$, and
$$|N'_i(x) \cap N'_i(y)|\ge \big( p_i - \alpha_i \big) p_i |Y_i|$$
for every $y \in X'$ and $i \in [r]$. Setting $Y'_i = N'_i(x)$ for each $i \in [r]$, it follows that
$$p_\ell\big( X', Y'_\ell \big) \ge p_\ell(X,Y_\ell) + \lambda \alpha_\ell \qquad \text{and} \qquad p_i\big( X', Y'_i \big) \ge p_i(X,Y_i) - \alpha_i$$
for every $i \in [r]$, as required. 
\end{proof}