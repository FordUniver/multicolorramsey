% In this file you should put the actual content of the blueprint.
% It will be used both by the web and the print version.
% It should *not* include the \begin{document}
%
% If you want to split the blueprint content into several files then
% the current file can be a simple sequence of \input. Otherwise It
% can start with a \section or \chapter for instance.

\begin{definition}
    \label{def:edge-colouring}
    \lean{SimpleGraph.EdgeColoring}
    \leanok
\end{definition}

\begin{definition}
    \label{def:colour-neighborhood}
    \uses{def:edge-colouring}
    Given an edge colouring, We write $N_i(u)$ to denote the neighbourhood of vertex $u$ in colour $i$. 
\end{definition}

%%%%%%%%%%%%%%%%%%%%%%%%%%%%%%%%%%%%%%%%%%%%%%%%%%%%%%%%%%%%%%%%%%%%%%%%%%%%%%%%%%%%%%%%%%%%%%%%%%%%
% prerequisite lemmata for the geometric lemma


\begin{lemma}\label{lem:moments}
    Let $U$ and\/ $U'$ be i.i.d.~random variables taking values in a finite set~$X$, and let\/ $\sigma_1,\ldots,\sigma_r \colon X \to \R^n$ be arbitrary functions. Then
    $$\Ex\Big[ \big\langle \sigma_1(U),\sigma_1(U') \big\rangle^{\ell_1} \cdots \big\langle \sigma_r(U),\sigma_r(U') \big\rangle^{\ell_r} \Big] \ge 0.$$
    for every $(\ell_1,\dots,\ell_r) \in \Z^r$ with $\ell_1,\dots,\ell_r \ge 0$.
    \end{lemma}

\begin{definition}\label{def:f}
    Let $r \in \N$.
    \begin{equation}\label{eq:f}
        f(x_1,\dots,x_r) = \sum_{j = 1}^r x_j \prod_{i \ne j} \big( 2 + \cosh\sqrt{x_i} \big),
        \end{equation}
    where we define $\cosh \sqrt{x}$ via its Taylor expansion
    $$\cosh\sqrt{x} = \sum_{n = 0}^\infty \frac{x^n}{(2n)!}.$$
    In particular, note that all of the coefficients of the Taylor expansion of $f$ are non-negative.
\end{definition}

\begin{lemma}\label{lem:special-function}
    \uses{def:f}
    Let $r \in \N$. The function $f \colon \R^r \to \R$ defined in~\eqref{eq:f} satisfies
    $$
    f(x_1,\dots,x_r) \le \left\{\begin{array}{cl}
    3^r r \exp\bigg( \displaystyle\sum_{i = 1}^r \sqrt{ x_i + 3r } \bigg) \quad & \text{if } \,\, x_i \ge - 3r \,\text{ for all }\, i \in [r];\\[+3ex]
    -1 & \text{otherwise.} 
    \end{array} \right.
    $$
\end{lemma}


%%%%%%%%%%%%%%%%%%%%%%%%%%%%%%%%%%%%%%%%%%%%%%%%%%%%%%%%%%%%%%%%%%%%%%%%%%%%%%%%%%%%%%%%%%%%%%%%%%%%
% the geometric lemma

\begin{definition}
    \label{def:p}
    \uses{def:colour-neighborhood}
    Let $r, n\in\N$. Given sets $X,Y \subset V(K_n)$ and a colour $i \in [r]$, define
    $$p_i(X,Y) = \min\bigg\{ \frac{|N_i(x) \cap Y|}{|Y|} : x \in X \bigg\},$$
\end{definition}

\begin{lemma}
    \label{lem:geometric}
    Let $r, n\in\N$. Set $\beta = 3^{-4r}$ and $C = 4r^{3/2}$.

    Let\/ $U$ and\/ $U'$ be i.i.d.~random variables taking values in a finite set~$X$, and let $\sigma_1,\ldots,\sigma_r \colon X \to \R^n$ be arbitrary functions. There exists $\lambda\ge-1$ and\/ $i\in[r]$ such that
    $$\Pr\Big( \big\langle \sigma_i(U),\sigma_i(U') \big\rangle \ge \lambda \, \text{ and } \, \big\langle \sigma_j(U), \sigma_j(U') \big\rangle \ge -1 \, \text{ for all } \, j \ne i \Big) \ge \beta e^{- C\sqrt{\lambda + 1}}.$$
\end{lemma}

\begin{proof}
    \uses{def:p, lem:special-function, lem:moments}
\end{proof}


%%%%%%%%%%%%%%%%%%%%%%%%%%%%%%%%%%%%%%%%%%%%%%%%%%%%%%%%%%%%%%%%%%%%%%%%%%%%%%%%%%%%%%%%%%%%%%%%%%%%
% the key lemma

\begin{lemma}
    \label{lem:key-lemma}
    \uses{def:p}

    Let $r, n\in\N$. Set $\beta = 3^{-4r}$ and $C = 4r^{3/2}$.

    Let\/ $\chi$ be an\/ $r$-colouring of\/ $E(K_n)$, let\/ $X,Y_1,\ldots,Y_r \subset V(K_n)$ be non-empty sets of vertices, and let $\alpha_1,\ldots,\alpha_r > 0$. There exists a vertex $x \in X$, a colour $\ell \in [r]$, sets $X' \subset X$ and\/ $Y'_1,\ldots,Y'_r\,$ with\/ $Y'_i \subset N_i(x) \cap Y_i\,$ for each $i \in [r]$, and\/ $\lambda \ge -1$, such that 
    \begin{equation}\label{eq:key:ell}
    |X'| \ge \beta e^{- C \sqrt{\lambda + 1}} |X| \qquad \text{and} \qquad p_\ell( X', Y'_\ell ) \ge p_\ell(X,Y_\ell) + \lambda \alpha_\ell,
    \end{equation}
    and moreover
    \begin{equation}\label{eq:key:alli}
    |Y'_i| = p_i(X,Y_i) |Y_i| \qquad \text{and} \qquad p_i( X', Y'_i ) \ge p_i(X,Y_i) - \alpha_i
    \end{equation}
    for every $i \in [r]$.
    \end{lemma}

\begin{proof}
    \uses{def:p, lem:geometric}
\end{proof}


%%%%%%%%%%%%%%%%%%%%%%%%%%%%%%%%%%%%%%%%%%%%%%%%%%%%%%%%%%%%%%%%%%%%%%%%%%%%%%%%%%%%%%%%%%%%%%%%%%%%
% the bound theorem

\begin{theorem}[Balister et al. 2024] \label{thm:Ramsey-multicolour}
    For each $r \ge 2$, there exists $\delta = \delta(r) > 0$ such that 
    %
    \begin{equation*}
        R_r(k) \le e^{-\delta k} r^{rk}
    \end{equation*} 
    %
    for all sufficiently large $k \in \N$. 
\end{theorem}

\begin{proof}
    \uses{lem:key-lemma}
    We will prove the theorem for $r = 2$.
\end{proof}
