% In this file you should put the actual content of the blueprint.
% It will be used both by the web and the print version.
% It should *not* include the \begin{document}
%
% If you want to split the blueprint content into several files then
% the current file can be a simple sequence of \input. Otherwise It
% can start with a \section or \chapter for instance.

\begin{definition}
  \label{def:edge-colouring}
  \lean{SimpleGraph.EdgeColoring}
  \leanok
  An edge colouring of a graph $G$ with colours $C$ is a map from the edge set $E(G)$ to $C$.
\end{definition}

\begin{definition}
  \label{def:colour-neighborhood}
  \uses{def:edge-colouring}
  \lean{SimpleGraph.EdgeColoring.coloredNeighborSet}
  \leanok
  Given an edge colouring, we write $N_i(u)$ to denote the neighbourhood of vertex $u$ in colour $i$.
\end{definition}

%%%%%%%%%%%%%%%%%%%%%%%%%%%%%%%%%%%%%%%%%%%%%%%%%%%%%%%%%%%%%%%%%%%%%%%%%%%%%%%%%%%%%%%%%%%%%%%%%%%%
% prerequisite lemmata for the geometric lemma

\begin{lemma}[Moments lemma (3.2)]
  \label{lem:moments}
  \lean{moments}
  Let $U$ and\/ $U'$ be i.i.d.~random variables taking values in a finite set~$X$, and let\/ $\sigma_1,\ldots,\sigma_r \colon X \to \R^n$ be arbitrary functions. Then
  $$\Ex\Big[ \big\langle \sigma_1(U),\sigma_1(U') \big\rangle^{\ell_1} \cdots \big\langle \sigma_r(U),\sigma_r(U') \big\rangle^{\ell_r} \Big] \ge 0.$$
  for every $(\ell_1,\dots,\ell_r) \in \Z^r$ with $\ell_1,\dots,\ell_r \ge 0$.
\end{lemma}
\begin{proof}
  %TODO
\end{proof}

\begin{definition}
  \label{def:coshsqrt}
  \lean{coshsqrt}
  \leanok
  Define $\cosh \sqrt{x}$ via its Taylor expansion
  $$\cosh\sqrt{x} = \sum_{n = 0}^\infty \frac{x^n}{(2n)!}.$$
\end{definition}

\begin{lemma}
  \label{lem:le-coshsqrt}
  \uses{def:coshsqrt}
  \lean{le_coshsqrt}
  For all $x \in \R$, it is $x \le 2 + \cosh\sqrt{x}$.
\end{lemma}
\begin{proof}
  %TODO
\end{proof}

\begin{lemma}
  \label{lem:coshsqrt-bd-pos}
  \uses{def:coshsqrt}
  $x \le 2 + \cosh \sqrt{x} \le 3 e^{\sqrt{x}}$ for every $x > 0$.
\end{lemma}
\begin{proof}
  %TODO
\end{proof}

\begin{lemma}
  \label{lem:coshsqrt-bd-neg}
  \uses{def:coshsqrt}
  $-1 \le \cosh \sqrt{x} \le 1$ for every $x < 0$.
\end{lemma}
\begin{proof}
  From the Taylor expansion, we get $\cosh \sqrt{x} = \cos \sqrt{-x}$.
\end{proof}

\begin{definition}
  \label{def:f}
  \uses{def:coshsqrt, lem:coshsqrt-bd-neg, lem:coshsqrt-bd-pos}
  \lean{f}
  Let $r \in \N$.
  \begin{equation}\label{eq:f}
    f(x_1,\dots,x_r) = \sum_{j = 1}^r x_j \prod_{i \ne j} \big( 2 + \cosh\sqrt{x_i} \big),
  \end{equation}
\end{definition}

\begin{lemma}
  \label{lem:taylor-nonneg}
  \uses{def:f}
  All of the coefficients of the Taylor expansion of $f$ are non-negative.
\end{lemma}
\begin{proof}
  %TODO
\end{proof}

\begin{lemma}[Special-function lemma, part 1 (3.3)]
  \label{lem:special-function-e}
  \uses{def:f}
  \lean{specialFunctionE}
  Let $r \in \N$. If $x_i \ge - 3r$ for all $i \in [r]$, the function $f \colon \R^r \to \R$ defined in~\eqref{eq:f} satisfies 
  $$ f(x_1,\dots,x_r) \le 3^r r \exp\bigg( \displaystyle\sum_{i = 1}^r \sqrt{ x_i + 3r } \bigg)$$
  \end{lemma}
  \begin{proof}
    %TODO
  \end{proof}

\begin{lemma}[Special-function lemma, part 2 (3.3)]
    \label{lem:special-function-ec}
    \uses{def:f}
    \lean{specialFunctionEc}
    Let $r \in \N$. If there exists an $i \in [r]$ with $x_i \le - 3r$, the function $f \colon \R^r \to \R$ defined in~\eqref{eq:f} satisfies
    $$
    f(x_1,\dots,x_r) \le -1$$
    \end{lemma}
    \begin{proof}
      %TODO
    \end{proof}

  %%%%%%%%%%%%%%%%%%%%%%%%%%%%%%%%%%%%%%%%%%%%%%%%%%%%%%%%%%%%%%%%%%%%%%%%%%%%%%%%%%%%%%%%%%%%%%%%%%%%
  % the geometric lemma

  \begin{definition}
    \label{def:p}
    \uses{def:colour-neighborhood}
    Let $r, n\in\N$. Given sets $X,Y \subset V(K_n)$ and a colour $i \in [r]$, define
    $$p_i(X,Y) = \min\bigg\{ \frac{|N_i(x) \cap Y|}{|Y|} : x \in X \bigg\},$$
  \end{definition}

  \begin{lemma}[Geometric lemma (3.1)]
    \label{lem:geometric}
    Let $r, n\in\N$. Set $\beta = 3^{-4r}$ and $C = 4r^{3/2}$.

    Let\/ $U$ and\/ $U'$ be i.i.d.~random variables taking values in a finite set~$X$, and let $\sigma_1,\ldots,\sigma_r \colon X \to \R^n$ be arbitrary functions. There exists $\lambda\ge-1$ and\/ $i\in[r]$ such that
    $$\Pr\Big( \big\langle \sigma_i(U),\sigma_i(U') \big\rangle \ge \lambda \, \text{ and } \, \big\langle \sigma_j(U), \sigma_j(U') \big\rangle \ge -1 \, \text{ for all } \, j \ne i \Big) \ge \beta e^{- C\sqrt{\lambda + 1}}.$$
  \end{lemma}

  \begin{proof}
    \uses{def:p, lem:special-function-e, lem:special-function-ec, lem:moments, def:f, lem:le-coshsqrt}
    For each $i \in [r]$, define $Z_i = 3r\big\langle \sigma_i(U),\sigma_i(U') \big\rangle$, and let $E$ be the event that $Z_i \ge -3r$ for every $i \in [r]$.

    Consider two cases:

    First assume $\Pr(E) \ge \beta$. Note that
    $$\Pr(E) = \Pr\Big( Z_i \ge - 3r \, \text{ for all } \, i \in [r] \Big),$$
    so with $\lambda = -1$,
    \begin{align*}
      \beta e^{-C\times 0} &= \beta\\
      &\le \Pr(E)\\
      &= \Pr\Big(3r\big\langle \sigma_i(U),\sigma_i(U') \big\rangle \ge -3r \text{ for all }i\Big)\\
      &= \Pr\Big( \big\langle \sigma_i(U),\sigma_i(U') \big\rangle \ge \lambda \, \text{ and } \, \big\langle \sigma_j(U), \sigma_j(U') \big\rangle \ge -1 \, \text{ for all } \, j \ne i \Big)
    \end{align*}
    hence the claimed inequality holds.

    Now, assume $\Pr(E) \le \beta$ and assume that for all $\lambda$,

    \begin{align}\label{eq:max:big:and:E:no}
      \Pr\left(E \cap \left( \bigcup_{i \in [r]} \left\{ \big\langle \sigma_i(U),\sigma_i(U') \big\rangle \ge \lambda \right\}\right)\right) < \beta r e^{-C\sqrt{\lambda + 1}}.
    \end{align}

    Observe that, since $x \le 2 + \cosh\sqrt{x}$ (Lemma~\ref{lem:le-coshsqrt}) and using Lemma~\ref{lem:moments} and linearity of expectation,
    \[
      \Ex\big[ f\big( Z_1,\ldots,Z_r \big) \big] = \Ex\left[ \sum_{j = 1}^r Z_j \prod_{i \ne j} \left( 2 + \cosh(\sqrt{Z_i}) \right) \right]
      \ge \sum_{j = 1}^r \Ex\left[ \prod_{i \in [r]} Z_i \right] = \sum_{j = 1}^r (3r)^r \Ex\left[ \prod_{i \in [r]} \big\langle \sigma_i(U),\sigma_i(U') \big\rangle \right] \ge 0
    \]

    We hence have
    \begin{align*}
      0 &\le \Ex\big[ f\big( Z_1,\ldots,Z_r \big) \big]\\
      &= \Ex\big[ f\big( Z_1,\ldots,Z_r \big)  \mathbf{1}_E \big] + \Ex\big[ f\big( Z_1,\ldots,Z_r \big) \mathbf{1}_{E^c} \big]\\
      &\le \Ex\left[ 3^r r \exp\bigg( \displaystyle\sum_{i = 1}^r \sqrt{ Z_i + 3r } \bigg)  \mathbf{1}_E \right] + \Ex\big[-1 \cdot \mathbf{1}_{E^c} \big]
    \end{align*}
    where we use Lemma~\ref{lem:special-function-e} to bound the case for $E$, and Lemma~\ref{lem:special-function-ec} for $E^c$.
    It follows that
    \begin{equation}\label{eq:eventE:inequality}
      1 - \Pr(E) \le 3^r r \cdot \Ex\bigg[ \exp\bigg( \sum_{i = 1}^r \sqrt{ Z_i + 3r } \bigg) \mathbf{1}_E \bigg].
    \end{equation}

    Let $M = \max \big\{ \big\langle \sigma_i(U),\sigma_i(U') \big\rangle : i \in [r] \big\}$. For any constant $c \le C - 1$, we have
    \begin{align*}
      \Ex\Big[ \exp\big( c \sqrt{M + 1} \big) \mathbf{1}_E \Big]
      & % = \int_{E} \exp\big( c \sqrt{M + 1}  \big) d\Pr
      \\
      & \le \, \Pr(E) + \int_{-1}^\infty \Pr\Big( \big\{ M \ge \lambda \big\} \cap E \Big) \cdot \frac{c}{2\sqrt{\lambda + 1}} \cdot e^{c \sqrt{\lambda + 1}} \,\mathrm{d}\lambda\\ %TODO not clear!
      & \le \, \beta + \beta r \int_{-1}^\infty \frac{c}{2\sqrt{\lambda + 1}} \cdot e^{- \sqrt{\lambda + 1}} \,\mathrm{d}\lambda\\
      & \le \, \beta (cr + 1),
    \end{align*}
    where in the first step we used Fubini's theorem, in the second we used $\Pr(E) \le \beta$ and $c \le C - 1$, and in the final step we used the fact that $\int_0^\infty \frac{1}{2\sqrt{x}} e^{-\sqrt{x}} \, \mathrm{d}x = 1$.

    In particular, applying this with $c = \sqrt{3}r^{3/2}$, and recalling that $C = 4r^{3/2} \ge c + 1$ and $\beta = 3^{-4r}$, it follows that
    \begin{align}
      1 - \Pr(E) &\le 3^r r \cdot \Ex\bigg[ \exp\bigg( \sum_{i = 1}^r \sqrt{ Z_i + 3r } \bigg) \mathbf{1}_E \bigg]\\
      &\le  3^r r \cdot \Ex\bigg[ \exp\bigg( r \cdot \max_{i \in [r]} \sqrt{ Z_i + 3r } \bigg) \mathbf{1}_E \bigg]\\
      &= 3^r r \cdot \Ex\bigg[ \exp\bigg( r \cdot \sqrt{ 3r} \cdot \sqrt{ (M + 1) } \bigg) \mathbf{1}_E \bigg]\\
      &\le 3^r r \cdot \beta (r^{2}\sqrt{3r} + 1) \le 1/3,
    \end{align}
    which contradicts our assumption that $\Pr(E) \le \beta$.

    Hence, there exists $\lambda \ge -1$ such that
    \begin{align}\label{eq:max:big:and:E}
      \beta r e^{-C\sqrt{\lambda + 1}} &\le \Pr\left(E \cap \left( \bigcup_{i \in [r]} \left\{ \big\langle \sigma_i(U),\sigma_i(U') \big\rangle \ge \lambda \right\}\right)   \right) \\
      &\le \sum_{i \in [r]} \Pr\left(E \cap \left\{ \big\langle \sigma_i(U),\sigma_i(U') \big\rangle \ge \lambda \right\} \right)\\
      &\le r \cdot \max_{i \in [r]} \Pr\left(E \cap \left\{ \big\langle \sigma_i(U),\sigma_i(U') \big\rangle \ge \lambda \right\} \right) ,
    \end{align}
    and therefore there exists an $i \in [r]$ as required.

  \end{proof}

  %%%%%%%%%%%%%%%%%%%%%%%%%%%%%%%%%%%%%%%%%%%%%%%%%%%%%%%%%%%%%%%%%%%%%%%%%%%%%%%%%%%%%%%%%%%%%%%%%%%%
  % the key lemma

  \begin{lemma}[Key lemma (2.2)]
    \label{lem:key-lemma}
    \uses{def:p}

    Let $r, n\in\N$. Set $\beta = 3^{-4r}$ and $C = 4r^{3/2}$.

    Let\/ $\chi$ be an\/ $r$-colouring of\/ $E(K_n)$, let\/ $X,Y_1,\ldots,Y_r \subset V(K_n)$ be non-empty sets of vertices, and let $\alpha_1,\ldots,\alpha_r > 0$. There exists a vertex $x \in X$, a colour $\ell \in [r]$, sets $X' \subset X$ and\/ $Y'_1,\ldots,Y'_r\,$ with\/ $Y'_i \subset N_i(x) \cap Y_i\,$ for each $i \in [r]$, and\/ $\lambda \ge -1$, such that
    \begin{equation}\label{eq:key:ell}
      \beta e^{- C \sqrt{\lambda + 1}} |X| \le |X'| \qquad \text{and} \qquad p_\ell(X,Y_\ell) + \lambda \alpha_\ell \le p_\ell( X', Y'_\ell ) ,
    \end{equation}
    and moreover
    \begin{equation}\label{eq:key:alli}
      |Y'_i| = p_i(X,Y_i) |Y_i| \qquad \text{and} \qquad  p_i(X,Y_i) - \alpha_i \le p_i( X', Y'_i )
    \end{equation}
    for every $i \in [r]$.
  \end{lemma}

  \begin{proof}
    \uses{def:p, lem:geometric}

    For each colour $i \in [r]$, define a function $\sigma_i \colon X \to \R^{Y_i}$ as follows: for each $x \in X$, choose a set $N'_i(x) \subset N_i(x) \cap Y_i$ of size exactly $p_i|Y_i|$, where $p_i = p_i(X,Y_i)$, and set
    $$\sigma_i(x) = \frac{\id_{N'_i(x)} - p_i\id_{Y_i}}{\sqrt{\alpha_ip_i|Y_i|}},$$
    where $\id_S \in \{0,1\}^{Y_i}$ denotes the indicator function of the set $S$. Note that, for any $x,y\in X$, %TODO why? mp direction suffices
    $$\lambda \le \big\langle \sigma_i(x),\sigma_i(y) \big\rangle \quad \Leftrightarrow \quad \big( p_i + \lambda\alpha_i \big) p_i |Y_i| \le |N'_i(x) \cap N'_i(y)|.$$
    Now, by Lemma~\ref{lem:geometric}, there exists $\lambda \ge -1$ and colour $\ell \in [r]$ such that
    $$\beta e^{- C\sqrt{\lambda + 1}} \le \Pr\Big( \lambda  \le \big\langle \sigma_\ell(U),\sigma_\ell(U') \big\rangle \, \text{ and } \, -1 \le \big\langle \sigma_i(U), \sigma_i(U') \big\rangle \, \text{ for all } \, i \ne \ell \Big) .$$
    where $U$, $U'$ are independent random variables distributed uniformly in the set~$X$. Hence there exists a vertex $x \in X$ and a set $X' \subset X$ (namely $\{x'\}$ for the vertex that exists for $U'$?) %TODO
    such that,
    $$\beta e^{- C \sqrt{\lambda + 1}} |X| \le |X'|$$ %TODO why?
    and
    $$\big( p_\ell + \lambda\alpha_\ell \big) p_\ell |Y_\ell | \le |N'_\ell(x) \cap N'_\ell(y)|$$
    for every $y \in X'$, and
    $$\big( p_i - \alpha_i \big) p_i |Y_i| \le |N'_i(x) \cap N'_i(y)|$$
    for every $y \in X'$ and $i \in [r]$.
    
    Setting $Y'_i = N'_i(x)$ for each $i \in [r]$, it follows that
    \begin{multline}
        \begin{aligned}
        p_\ell(X,Y_\ell) + \lambda \alpha_\ell &= \frac{ \big( p_\ell + \lambda\alpha_\ell \big) p_\ell |Y_\ell |}{p_\ell |Y_\ell|}\\
        &= \frac{ \big( p_\ell + \lambda\alpha_\ell \big) p_\ell |Y_\ell |}{| N'_\ell (x)|}\\
        &= \min\bigg\{ \frac{ \big( p_\ell + \lambda\alpha_\ell \big) p_\ell |Y_\ell |}{| N'_\ell (x)|} : x' \in X' \bigg\}\\
        &\le \min\bigg\{ \frac{|N'_\ell(x') \cap  N'_\ell (x)|}{| N'_\ell (x)|} : x' \in X' \bigg\}\\
        &= \min\bigg\{ \frac{|N_\ell(x') \cap  N'_\ell (x)|}{| N'_\ell (x)|} : x' \in X' \bigg\}\\
        &= p_\ell\big( X', N'_\ell (x) \big) = p_\ell\big( X', Y'_\ell \big) \\
      \end{aligned}
    \end{multline}

    and $$  p_i(X,Y_i) - \alpha_i \le \qquad p_i\big( X', Y'_i \big) $$
    for every $i \in [r]$, as required.
  \end{proof}

  %%%%%%%%%%%%%%%%%%%%%%%%%%%%%%%%%%%%%%%%%%%%%%%%%%%%%%%%%%%%%%%%%%%%%%%%%%%%%%%%%%%%%%%%%%%%%%%%%%%%
  % the bound theorem

  \begin{theorem}[Balister et al. 2024]
    \label{thm:Ramsey-multicolour}
    For each $r \ge 2$, there exists $\delta = \delta(r) > 0$ such that
    %
    \begin{equation*}
      R_r(k) \le e^{-\delta k} r^{rk}
    \end{equation*}
    %
    for all sufficiently large $k \in \N$.
  \end{theorem}

  \begin{proof}
    \uses{lem:key-lemma}
    We will prove the theorem for $r = 2$.
    %TODO
  \end{proof}
