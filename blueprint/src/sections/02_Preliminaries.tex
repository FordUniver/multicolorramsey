% Preliminaries section - basic definitions only
% This section contains the fundamental definitions needed for the formalization

\section{Preliminaries}


\subsection{Elementary combinatorics}

\begin{lemma}[Multinomial coefficient bound]
  \label{lem:multibounds} % Source paper uses same label
  \paperref{Lemma A.1}
  %
  For each \(k,t,r \in \N\) with \(3 \le t \le k\), we have
  %
  \begin{equation*}
    \binom{rk-t}{k,\dots,k,k-t} \le e^{-(r-1)t^2/3rk} r^{rk-t}.
  \end{equation*}
  %
\end{lemma}
%
\begin{proof}
  To prove this, observe that 
  %
  \begin{equation*}
    \binom{rk-t}{k,\dots,k,k-t} \binom{rk}{k,\dots,k}^{-1} = \, \prod_{i = 0}^{t - 1} \frac{k - i}{rk - i} = r^{-t} \,\prod_{i = 0}^{t-1} \bigg( 1 - \frac{(r-1)i}{rk - i} \bigg) \le r^{-t} \cdot e^{-(r-1)t^2/3rk}.
  \end{equation*}
  %
  Since \(\binom{rk}{k,\dots,k} \le r^{rk}\), the claimed bound follows.
\end{proof}


\subsection{Graphs and colourings}

\begin{definition}[Complete graph]
  \label{def:complete-graph}
  \lean{CompleteGraph}
  \leanok
\end{definition}

\begin{definition}[Edge colouring]
  \label{def:edge-colouring}
  \uses{def:complete-graph}
  \lean{SimpleGraph.EdgeColoring}
  \leanok
  %
  An edge colouring of a graph \(G\) with colours \(C\) is a map from the edge set \(E(G)\) to \(C\).
\end{definition}

\begin{definition}[Colour neighbourhood]
  \label{def:colour-neighborhood}
  \uses{def:edge-colouring}
  \lean{SimpleGraph.EdgeColoring.coloredNeighborSet}
  \leanok
  %
  Given an edge colouring, we write \(N_i(u)\) to denote the neighbourhood of vertex \(u\) in colour \(i\).
\end{definition}

\begin{definition}[Function \(p_i(X,Y)\)]
  \label{def:p}
  \uses{def:colour-neighborhood}
  \lean{p}
  %
  Let \(r, n\in\N\). Given sets \(X,Y \subset V(K_n)\) and a colour \(i \in [r]\), define
  %
  \begin{equation*}
    p_i(X,Y) = \min\bigg\{ \frac{|N_i(x) \cap Y|}{|Y|} : x \in X \bigg\},
  \end{equation*}
  %
\end{definition}

% [DEFINITION] Cliques and monochromatic structures
% Source: blueprint/paper/sections/section_introduction.tex line 6 (monochromatic clique K_k)
% No deprecated blueprint equivalent
% Lean status: ✗ Not implemented
% Label: def:monochromatic-clique
% Dependencies: def:complete-graph, def:edge-colouring
% TODO: Add definition of monochromatic clique in edge coloring

\begin{definition}[Monochromatic cliques]
  \label{def:monochromatic-clique}
  \uses{def:complete-graph, def:edge-colouring}
  % Source paper defines monochromatic clique K_k in edge coloring
\end{definition}

% [DEFINITION] Ramsey numbers R_r(k)
% Source: blueprint/paper/sections/section_introduction.tex line 3-4 (informal definition)
% No deprecated blueprint equivalent
% Lean status: ✗ Not implemented
% Label: def:ramsey-number
% Dependencies: def:complete-graph, def:edge-colouring, def:monochromatic-clique
% TODO: Add definition R_r(k) = min{n : every r-coloring of E(K_n) contains monochromatic K_k}

\begin{definition}[Multicolor Ramsey numbers]
  \label{def:ramsey-number}
  \uses{def:complete-graph, def:edge-colouring, def:monochromatic-clique}
  % Source paper defines R_r(k) = min{n : every r-coloring of E(K_n) contains monochromatic K_k}
\end{definition}

% [DEFINITION] Off-diagonal Ramsey numbers R_r(k_1,...,k_r)
% Source: blueprint/paper/sections/section_the_book_theorem.tex line 7 (R_r(k,...,k,k-t))
% No deprecated blueprint equivalent
% Lean status: ✗ Not implemented
% Label: def:ramsey-number-off-diagonal
% Dependencies: def:ramsey-number
% TODO: Add definition R_r(k_1,...,k_r) for off-diagonal Ramsey numbers

\begin{definition}[Off-diagonal Ramsey numbers]
  \label{def:ramsey-number-off-diagonal}
  \uses{def:ramsey-number}
  % Source paper uses R_r(k_1,...,k_r) for off-diagonal Ramsey numbers
\end{definition}

% [DEFINITION] Books (A,B)
% Source: blueprint/paper/sections/section_the_book_theorem.tex line 3-5 (book definition)
% No deprecated blueprint equivalent
% Lean status: ✗ Not implemented
% Label: def:book
% Dependencies: def:complete-graph
% TODO: Add definition of book graph: vertex set A ∪ B, edge set {uv : {u,v} ⊄ B}
% TODO: Add definition of (t,m)-book with |A| = t and |B| = m

\begin{definition}[Books]
  \label{def:book}
  \uses{def:complete-graph}
  % Source paper defines book graph with vertex set A ∪ B, edge set {uv : {u,v} ⊄ B}
  % A (t,m)-book has |A| = t and |B| = m
\end{definition}


\subsection{Trigonometry}

\begin{definition}[cosh√x function]
  \label{def:coshsqrt} % Source paper uses same label
  \lean{coshsqrt}
  \leanok
  %
  Define \(\cosh \sqrt{x}\) via its Taylor expansion
  %
  \begin{equation*}
    \cosh\sqrt{x} = \sum_{n = 0}^\infty \frac{x^n}{(2n)!}.
  \end{equation*}
  %
\end{definition}

\begin{lemma}[Bounds on \(\cosh\sqrt{x}\) for positive arguments]
  \label{lem:coshsqrt-bd-pos}
  \uses{def:coshsqrt}
  \lean{lt_coshsqrt}
  \leanok
  %
  \(x \le 2 + \cosh \sqrt{x} \le 3 e^{\sqrt{x}}\) for every \(x > 0\).
\end{lemma}
%
\begin{proof}
  For the lower-bound, using the fact that \(x>0\) and all coefficients of \(\cosh \sqrt{x}\) are positive,
  %
  \begin{equation*}
    2+\cosh \sqrt{x} - x \ge 2 - \frac{x}{2} + \frac{x^2}{24} = \frac{1}{2}+\frac{1}{24}(x-6)^2 \ge 0.
  \end{equation*}
  %
  For the upper bound, we observe that because by comparing coefficients,
  %
  \begin{equation*}
    e^{\sqrt{x}} = \sum 3\frac{x^{n/2}}{n!}  \ge \cosh\sqrt{x}
  \end{equation*}
  %
  Finally, \(2e^{\sqrt{x}}\ge 2\) when \(x>0\).
\end{proof}

\begin{lemma}[Bounds on \(\cosh\sqrt{x}\) for negative arguments]
  \label{lem:coshsqrt-bd-neg}
  \uses{def:coshsqrt}
  \lean{icc_coshsqrt_neg}
  \leanok
  %
  \(-1 \le \cosh \sqrt{x} \le 1\) for every \(x < 0\).
\end{lemma}
%
\begin{proof}
  From the Taylor expansion, we get \(\cosh \sqrt{x} = \cos \sqrt{-x}\).
\end{proof}

\begin{lemma}[Lower bound for cosh√x]
  \label{lem:le-coshsqrt} % Source paper uses same label
  \uses{def:coshsqrt, lem:coshsqrt-bd-neg}
  \lean{one_le_coshsqrt}
  \leanok
  %
  For all \(x \in \R\), it is \(1 \le 2 + \cosh\sqrt{x}\).
\end{lemma}
%
\begin{proof}
  When \(x\) is negative, we use \(\cosh \sqrt{x} = \cos \sqrt{-x}\ge -1\). When \(x\) is positive this is implied by the fact that all coefficients in the power series of \(\cosh \sqrt{x}\) are positive.
\end{proof}

\begin{definition}
  \label{def:f}
  \uses{def:coshsqrt, lem:coshsqrt-bd-neg, lem:coshsqrt-bd-pos}
  \lean{f}
  \leanok
  %
  Let \(r \in \N\).
  %
  \begin{equation}\label{eq:f}
    f(x_1,\dots,x_r) = \sum_{j = 1}^r x_j \prod_{i \ne j} \big( 2 + \cosh\sqrt{x_i} \big).
  \end{equation}
  %
\end{definition}

\begin{lemma}
  \label{lem:taylor-nonneg}
  \uses{def:f}
  %
  All of the coefficients of the Taylor expansion of \(f\) are non-negative.
\end{lemma}
%
\begin{proof}
  By definition \(2+\cosh\sqrt{x}\) has positive coefficients. It suffices to observe that the product and sum of two Taylor series with only positive coefficients has again positive coefficients. Since \(f\) is such a combination of sums and products of Taylor series with only positive coefficients, we get the claim.
\end{proof}


\subsection{Integration theory}

% [LEMMA] Fubini theorem for finite measures
% Source: blueprint/paper/sections/section_the_proof_of_the_key_lemma.tex (geometric lemma proof line 100-104)
% No deprecated blueprint equivalent
% Lean status: ✓ Uses Mathlib Fubini theorem
% Label: lem:fubini-finite
% Dependencies: def:uniform-product
% TODO: Add Fubini theorem application for expectation computation

\begin{lemma}[Fubini for finite measures]
  \label{lem:fubini-finite}
  \uses{}
  % Source paper uses Fubini in geometric lemma proof
\end{lemma}

% [LEMMA] Exponential function estimates
% Source: blueprint/paper/sections/section_the_proof_of_the_key_lemma.tex line 102-104 (integration bounds)
% No deprecated blueprint equivalent
% Lean status: ✓ Uses standard exponential estimates
% Label: lem:exp-estimates
% Dependencies: None (elementary analysis)
% TODO: Add bounds for integrals involving exponentials

\begin{lemma}[Exponential estimates]
  \label{lem:exp-estimates}
  % Source paper uses exponential decay estimates
\end{lemma}

% [LEMMA] Basic integral bounds for exponential decay
% Source: blueprint/paper/sections/section_the_proof_of_the_key_lemma.tex (geometric lemma proof)
% No deprecated blueprint equivalent
% Lean status: ✓ Implemented as `integral_rpow_mul_exp_neg_rpow_le` in Integrals.lean
% Label: lem:integral-exp-decay
% Dependencies: lem:fubini-finite, lem:exp-estimates
% TODO: Add ∫₀^∞ x^a e^{-b√x} dx bounds for geometric lemma integration

\begin{lemma}[Exponential decay integrals]
  \label{lem:integral-exp-decay}
  \uses{lem:fubini-finite, lem:exp-estimates}
  \lean{integral_rpow_mul_exp_neg_rpow_le}
  % Source paper uses these bounds implicitly in geometric lemma proof
\end{lemma}

% [LEMMA] Integration by parts for special functions
% Source: blueprint/paper/sections/section_the_proof_of_the_key_lemma.tex (geometric lemma proof)
% No deprecated blueprint equivalent
% Lean status: ✓ Partially implemented in Integrals.lean
% Label: lem:integration-by-parts-special
% Dependencies: lem:integral-bounds
% TODO: Add integration techniques for ∫ f(x₁,...,xᵣ) dx₁...dxᵣ type integrals

\begin{lemma}[Integration by parts for special functions]
  \label{lem:integration-by-parts-special}
  \uses{lem:integral-bounds}
  % Source paper uses integration by parts implicitly
\end{lemma}

% [LEMMA] Substitution formulas for geometric lemma
% Source: blueprint/paper/sections/section_the_proof_of_the_key_lemma.tex (geometric lemma proof)
% No deprecated blueprint equivalent
% Lean status: ✓ Used in GeometricLemma.lean proof
% Label: lem:substitution-geometric
% Dependencies: lem:fubini-finite
% TODO: Add change of variables for probability density calculations

\begin{lemma}[Substitution formulas]
  \label{lem:substitution-geometric}
  \uses{lem:fubini-finite}
  % Source paper uses substitution in geometric proof
\end{lemma}