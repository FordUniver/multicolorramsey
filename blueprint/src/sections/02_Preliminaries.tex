\subsection{Graphs and colourings}

\begin{definition}
  \label{def:edge-colouring}
  \lean{SimpleGraph.EdgeColoring}
  \leanok
  An edge colouring of a graph $G$ with colours $C$ is a map from the edge set $E(G)$ to $C$.
\end{definition}

\begin{definition}
  \label{def:colour-neighborhood}
  \uses{def:edge-colouring}
  \lean{SimpleGraph.EdgeColoring.coloredNeighborSet}
  \leanok
  Given an edge colouring, we write $N_i(u)$ to denote the neighbourhood of vertex $u$ in colour $i$.
\end{definition}

\begin{definition}
  \label{def:p}
  \uses{def:colour-neighborhood}
  \lean{p}
  Let $r, n\in\N$. Given sets $X,Y \subset V(K_n)$ and a colour $i \in [r]$, define
  $$p_i(X,Y) = \min\bigg\{ \frac{|N_i(x) \cap Y|}{|Y|} : x \in X \bigg\},$$
\end{definition}


%---------------------------------------------------------------------------------------------------
\subsection{Multinomial coefficients}

\begin{lemma}[Lemma A.1 in Balister et al.]  
  \label{lem:multibounds}
  For each $k,t,r \in \N$ with $3 \le t \le k$, we have
  $$\binom{rk-t}{k,\dots,k,k-t} \le e^{-(r-1)t^2/3rk} r^{rk-t}.$$
\end{lemma}
  
\begin{proof}
  To prove this, observe that 
  $$\binom{rk-t}{k,\dots,k,k-t} \binom{rk}{k,\dots,k}^{-1} = \, \prod_{i = 0}^{t - 1} \frac{k - i}{rk - i} = r^{-t} \,\prod_{i = 0}^{t-1} \bigg( 1 - \frac{(r-1)i}{rk - i} \bigg) \le r^{-t} \cdot e^{-(r-1)t^2/3rk}.$$
  Since $\binom{rk}{k,\dots,k} \le r^{rk}$, the claimed bound follows.
\end{proof}


%---------------------------------------------------------------------------------------------------
\subsection{Trigonometry}

\begin{definition}
  \label{def:coshsqrt}
  \lean{coshsqrt}
  \leanok
  Define $\cosh \sqrt{x}$ via its Taylor expansion
  $$\cosh\sqrt{x} = \sum_{n = 0}^\infty \frac{x^n}{(2n)!}.$$
\end{definition}

\begin{lemma}
  \label{lem:le-coshsqrt}
  \uses{def:coshsqrt}
  \lean{one_le_coshsqrt}
  \leanok
  For all $x \in \R$, it is $1 \le 2 + \cosh\sqrt{x}$.
\end{lemma}
\begin{proof}
  When $x$ is negative, we use $\cosh \sqrt{x} = \cos \sqrt{-x}\ge -1$. When $x$ is positive this is implied by the fact that all coefficients in the power series of $\cosh \sqrt{x}$ are positive.
\end{proof}

\begin{lemma}
  \label{lem:coshsqrt-bd-pos}
  \uses{def:coshsqrt}
  \lean{lt_coshsqrt}
  \leanok
  $x \le 2 + \cosh \sqrt{x} \le 3 e^{\sqrt{x}}$ for every $x > 0$.
\end{lemma}
\begin{proof}
  For the lower-bound, using the fact that $x>0$ and all coefficients of $\cosh \sqrt{x}$ are positive,
  \begin{equation*}
    2+\cosh \sqrt{x} - x \ge 2 - \frac{x}{2} + \frac{x^2}{24} = \frac{1}{2}+\frac{1}{24}(x-6)^2 \ge 0.
\end{equation*}
For the upper bound, we observe that because by comparing coefficients,
\begin{equation*}
  e^{\sqrt{x}} = \sum 3\frac{x^{n/2}}{n!}  \ge \cosh\sqrt{x}
\end{equation*}
Finally, $2e^{\sqrt{x}}\ge 2$ when $x>0$.
\end{proof}

\begin{lemma}
  \label{lem:coshsqrt-bd-neg}
  \uses{def:coshsqrt}
  \lean{icc_coshsqrt_neg}
  \leanok
  $-1 \le \cosh \sqrt{x} \le 1$ for every $x < 0$.
\end{lemma}
\begin{proof}
  From the Taylor expansion, we get $\cosh \sqrt{x} = \cos \sqrt{-x}$.
\end{proof}

\begin{definition}
  \label{def:f}
  \uses{def:coshsqrt, lem:coshsqrt-bd-neg, lem:coshsqrt-bd-pos}
  \lean{f}
  \leanok
  Let $r \in \N$.
  \begin{equation}\label{eq:f}
    f(x_1,\dots,x_r) = \sum_{j = 1}^r x_j \prod_{i \ne j} \big( 2 + \cosh\sqrt{x_i} \big).
  \end{equation}
\end{definition}

\begin{lemma}
  \label{lem:taylor-nonneg}
  \uses{def:f}
  All of the coefficients of the Taylor expansion of $f$ are non-negative.
\end{lemma}
\begin{proof}
  By definition $2+\cosh\sqrt{x}$ has positive coefficients. It suffices to observe that the product and sum of two Taylor series with only positive coefficients has again positive coefficients. Since $f$ is such a combination of sums and products of Taylor series with only positive coefficients, we get the claim.
\end{proof}