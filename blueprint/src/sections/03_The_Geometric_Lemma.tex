% The Geometric Lemma section - moments lemma, geometric lemma, and key lemma proof
% This section contains the geometric tools needed to prove the key lemma

\section{The Geometric Lemma}

% ============================================================================
% MOMENTS LEMMA
% ============================================================================

\begin{lemma}[Moments lemma]
  \label{lem:moments} % Source paper uses same label
  \paperref{Lemma 3.2}
  \uses{}
  \lean{moments}
  %
  Let \(U\) and \(U'\) be i.i.d.~random variables taking values in a finite set~\(X\), and let \(\sigma_1,\ldots,\sigma_r \colon X \to \R^n\) be arbitrary functions. Then
  %
  \begin{equation*}
    \Ex\Big[ \big\langle \sigma_1(U),\sigma_1(U') \big\rangle^{\ell_1} \cdots \big\langle \sigma_r(U),\sigma_r(U') \big\rangle^{\ell_r} \Big] \ge 0.
  \end{equation*}
  %
  for every \((\ell_1,\dots,\ell_r) \in \Z^r\) with \(\ell_1,\dots,\ell_r \ge 0\).
\end{lemma}
%
\begin{proof}
  To simplify the notation, let us write
	%
	\begin{equation*}
		\big\langle \sigma_1(U),\sigma_1(U') \big\rangle^{\ell_1} \cdots \big\langle \sigma_r(U), \sigma_r(U') \big\rangle^{\ell_r} = \prod_{i = 1}^\ell \big\langle \sigma_{a_i}(U), \sigma_{a_i}(U') \big\rangle
	\end{equation*}
	%
	where \(\ell = \ell_1 + \cdots + \ell_r\) and \((a_1,\dots,a_\ell)\) is such that \(\big| \big\{ i \in [\ell] : a_i = j \big\} \big| = \ell_j\) for each \(j \in [r]\). Now set
	%
	\begin{equation*}
		Z = \sigma_{a_1}(U) \otimes \sigma_{a_2}(U) \otimes \cdots \otimes \sigma_{a_\ell}(U) \quad \text{and} \quad Z' = \sigma_{a_1}(U') \otimes \sigma_{a_2}(U') \otimes \cdots \otimes \sigma_{a_\ell}(U'),
	\end{equation*}
	%
	and note that
	%
	\begin{equation*}
		\big\langle Z, Z' \big\rangle = \prod_{i = 1}^\ell \big\langle \sigma_{a_i}(U), \sigma_{a_i}(U') \big\rangle,
	\end{equation*}
	%
	since \(\langle u_1 \otimes \cdots \otimes u_r, v_1 \otimes \cdots \otimes v_r \rangle = \langle u_1, v_1 \rangle  \cdots \langle v_r , u_r \rangle \) for any vectors \(u_i,v_i \in \R^n\). Finally, note that \(Z\) and \(Z'\) are independent and identically distributed random vectors, and therefore
	%
	\begin{equation*}
		\Ex \big[ \langle Z, Z' \rangle \big] = \Ex_{Z'} \big[ \Ex_Z \big[ \langle Z, Z' \rangle \big] \big] =
		\Ex_{Z'} \big[ \big\langle \Ex[Z], Z' \big\rangle \big] = \big\langle \Ex[Z], \Ex[Z'] \big\rangle \ge 0,
	\end{equation*}
	%
	as required, where the final inequality holds because \(\Ex[Z] = \Ex[Z']\).
\end{proof}

\begin{remark}
  Let \(g(x):= cosh(\sqrt{x})\) where \(cosh(x)\) is defined by \(cosh(\sqrt{x})=\overunderset{\infty}{n=0}{\sum}\frac{x^n}{(2n)!}\), then: 
  %
  \begin{enumerate}
    \item For all \(x\geq 0\), we have \(x \leq 2+cosh(\sqrt{x}) \leq 3e^{\sqrt{x}}\).
    \item For all \(x <0\), we have \(-1 \leq cosh(\sqrt{x})\leq 1\). In particular, \(2+cosh(x)\geq 1\) for all \(x\in \mathbb{R}\).
  \end{enumerate}
  %
\end{remark}
%
\begin{proof}
  To prove (1), let \(x\geq 0\) then:
  %
  \begin{equation*}
    2+cosh(\sqrt{x}) = 3 + \overunderset{\infty}{n=1}{\sum}\frac{x^n}{(2n)!}\geq 3 + \frac{x}{2}+\frac{x^2}{24}.
  \end{equation*}
  %
  However,
  %
  \begin{equation*}
    3+\frac{x}{2}+\frac{x^2}{24}\geq x \iff \frac{1}{24}(x^2-12x+72)\geq 0 \iff \frac{1}{24}((x-6)^2+36)\geq 0,s
  \end{equation*}
  %
  which concludes the lower bound in the first point. To conclude (1), we note the following for \(x\geq0\): 
  %
  \begin{equation*}
    3e^{\sqrt{x}}= 3\overunderset{\infty}{n=0}{\sum}\frac{x^{n/2}}{n!}\geq 3\overunderset{\infty}{n=0}{\sum}\frac{x^{n}}{(2n)!} \geq 2 + \overunderset{\infty}{n=0}{\sum}\frac{x^{n}}{(2n)!}= 2 + cosh(\sqrt{x}).
  \end{equation*}
  %
  To prove (2), note that for \(x<0\),
  %
  \begin{equation*}
    cosh(\sqrt{x})= \overunderset{\infty}{n=0}{\sum}\frac{x^n}{(2n)!}= \overunderset{\infty}{n=0}{\sum}\frac{(-1)^n \sqrt{-x}^{2n}}{(2n)!}= cos(\sqrt{-x}).
  \end{equation*}
  %
  (2) then follows by noting that \(cos(\sqrt{-x}) \in [-1,1]\) for all \(x<0\). 
\end{proof}

% ============================================================================
% SPECIAL FUNCTION LEMMAS
% ============================================================================

\begin{lemma}[Special function upper bound]
  \label{lem:special:function:1} % Source paper: f(x) ≤ 3ʳr exp(∑ᵢ √(xᵢ+3r)) if xᵢ ≥ -3r ∀i
  \paperref{Lemma 3.3}
  \uses{def:f, lem:coshsqrt-bd-pos, lem:taylor-nonneg}
  \lean{specialFunctionE}
  \leanok
  %
  Let \(r \in \N\). If \(x_i \ge - 3r\) for all \(i \in [r]\), the function \(f \colon \R^r \to \R\) defined in~\eqref{eq:f} satisfies 
  %
  \begin{equation*}
     f(x_1,\dots,x_r) \le 3^r r \exp\bigg( \displaystyle\sum_{i = 1}^r \sqrt{ x_i + 3r } \bigg)
  \end{equation*}
  %
\end{lemma}
%
\begin{proof} 
  %
  \begin{equation*}
    f(x_1,\dots,x_r)= \Bigl(\overunderset{r}{i=1}{\prod}(2+cosh\sqrt{x_i})\Bigr)\overunderset{r}{j=1}{\sum}\frac{x_j}{2+cosh \sqrt{x_j}}
     \overset{**}{\leq}r\Bigl(\overunderset{r}{i=1}{\prod} 3e^{\sqrt{x_i+3r}}\Bigr),
  \end{equation*}
  %
  % justification of inequality (**)
  % bounding the factors with x_i>0
  To see (**), we write down upper bounds for all the factors of the LHS of (**).
  First, for every \(i \in [r]\), such that \(x_i \geq 0\), we note that \(cosh(\sqrt{x_i} )\leq cosh(\sqrt{x_i+3r})\) (which follows since all
  the coefficients in the taylor expansion of \(cosh(\sqrt{y}))\) are non-negative and thus \(cosh(\sqrt{y})\) is non-decreasing on the positive reals.
  In turn, \(2+cosh(\sqrt{x_i+3r}) \leq 3e^{\sqrt{x_i+3r}}\) by point 1 of the above remark. We conculde that if \(x_i \geq 0\), 
  \(2 + cosh{\sqrt{x_i}} \leq 3 e^{\sqrt{x_i+3r}}\).

  % bounding the factors with x_i<0
  Next, we condiser the indices \(i \in [r]\) such that \(x_i<0\). In this case \begin{equation*}
    2+cosh(\sqrt{x_i})= 2 + cos(\sqrt{-x_i}) \leq 3 \leq 3 e^{\sqrt{x_i+3r}},
  \end{equation*} 
  since \(x_i \geq -3r\).
  %justification of the sum factor
  It remains to show that \(\frac{x_j}{2+cosh(\sqrt{x_j})} \leq 1\) for every \(j \in [r]\). If \(x_j\geq 0\) this follows directly from point (1) in the above remark.
  If \(x_j<0\), note that the numerator of the fraction is negative while the denominator is a positive number between 1 and 3 by point(2) of the remark above.
\end{proof}

\begin{lemma}[Special function lower bound]
  \label{lem:special:function:2} % Source paper: f(x) ≤ -1 if ∃i: xᵢ ≤ -3r
  \paperref{Lemma 3.3}
  \uses{def:f, lem:coshsqrt-bd-neg}
  \lean{specialFunctionEc}
  \leanok
  %
  Let \(r \in \N\). If there exists an \(i \in [r]\) with \(x_i \le - 3r\), the function \(f \colon \R^r \to \R\) defined in~\eqref{eq:f} satisfies
  %
  \begin{equation*}
    f(x_1,\dots,x_r) \le -1
  \end{equation*}
  %
\end{lemma}
%
\begin{proof}
  Let \(i \in [r]\) satisfy \(x_i \leq -3r\). Note that if \(x\geq 0\), then \(\frac{x}{2+cosh(\sqrt{x})}\leq 1\) by point (1) of the above remark.
  Trivially, we have \(\frac{x}{2+cosh{\sqrt{x}}}<0\leq 1\) for \(x<0\) (note that \(2+cosh(\sqrt{x})\geq 1\).)
  Therefore, we have 
  %
  \begin{equation*}
    \underset{j \in [r], j \neq i}{\sum} \frac{x_j}{2+cosh(\sqrt{x_j})} \leq r-1.
  \end{equation*}
  %
  On the other hand, we know that \(\frac{x_i}{2+cosh(\sqrt{x_i})} \leq \frac{x_i}{3}\) by part (2) in the remark above and therefore,
  %
  \begin{equation*}
    \overunderset{r}{j=1}{\sum}\frac{x_j}{2+cosh(\sqrt{x_j})}\leq \frac{x_i}{3}+r-1\leq -1.
  \end{equation*}
  %
  Finally we write 
  %
  \begin{equation*}
    f(x_1,\dots,x_r) = \Bigl(\overunderset{r}{i=1}{\prod}(2+cosh\sqrt{x_i})\Bigr)\overunderset{r}{j=1}{\sum}\frac{x_j}{2+cosh \sqrt{x_j}} \leq -1,
  \end{equation*}
  %
  since the sum in the last term is \(\leq -1\) while the product is at least \(1\) (each factor in the product is at least 1).
\end{proof}

\begin{remark}
	The key idea in the two lemmas above was to find an entire function, \(2 + \cosh\sqrt{x}\), which \((a)\) has a Taylor expansion at \(x = 0\) with non-negative coefficients; \((b)\) is bounded on the negative real axis; and \((c)\) does not grow too quickly on the positive axis. The Phragm\'en--Lindel\"of theorem
	implies that functions satisfying \((a)\) and \((b)\) must grow at least as fast as \(\exp\big( \Omega(\sqrt{x}) \big)\) on the positive real axis.  Thus the bound on the growth of \(f\) given by the lemma is essentially best possible for constructions of this type.
\end{remark}

% ============================================================================
% GEOMETRIC LEMMA
% ============================================================================

\begin{lemma}[Geometric lemma]
  \label{lem:geometric} % Source paper uses label lem:lambda
  \paperref{Lemma 3.1}
  \uses{lem:moments, lem:special:function:1, lem:special:function:2, lem:taylor-nonneg, lem:integral-exp-decay}
  \lean{geometric}
  %
  Let \(r, n\in\N\). Set \(\beta = 3^{-4r}\) and \(C = 4r^{3/2}\).
  Let \(U\) and \(U'\) be i.i.d.~random variables taking values in a finite set~\(X\), and let \(\sigma_1,\ldots,\sigma_r \colon X \to \R^n\) be arbitrary functions. There exists \(\lambda\ge-1\) and \(i\in[r]\) such that
  %
  \begin{equation*}
    \Pr\Big( \big\langle \sigma_i(U),\sigma_i(U') \big\rangle \ge \lambda \, \text{ and } \, \big\langle \sigma_j(U), \sigma_j(U') \big\rangle \ge -1 \, \text{ for all } \, j \ne i \Big) \ge \beta e^{- C\sqrt{\lambda + 1}}.
  \end{equation*}
  %
\end{lemma}

\begin{proof}
  \uses{def:p, lem:special:function:1, lem:special:function:2, lem:moments, def:f, lem:le-coshsqrt}
  For each \(i \in [r]\), define \(Z_i = 3r\big\langle \sigma_i(U),\sigma_i(U') \big\rangle\), and let \(E\) be the event that \(Z_i \ge -3r\) for every \(i \in [r]\).

  Consider two cases:

  First assume \(\Pr(E) \ge \beta\). Note that
  %
  \begin{equation*}
    \Pr(E) = \Pr\Big( Z_i \ge - 3r \, \text{ for all } \, i \in [r] \Big),
  \end{equation*}
  %
  so with \(\lambda = -1\),
  %
  \begin{align*}
    \beta e^{-C\times 0} &= \beta\\
    &\le \Pr(E)\\
    &= \Pr\Big(3r\big\langle \sigma_i(U),\sigma_i(U') \big\rangle \ge -3r \text{ for all }i\Big)\\
    &= \Pr\Big( \big\langle \sigma_i(U),\sigma_i(U') \big\rangle \ge \lambda \, \text{ and } \, \big\langle \sigma_j(U), \sigma_j(U') \big\rangle \ge -1 \, \text{ for all } \, j \ne i \Big)
  \end{align*}
  %
  hence the claimed inequality holds.

  Now, assume \(\Pr(E) \le \beta\) and assume for the sake of a contradiction that for all \(\lambda \geq -1\),

  \begin{align}\label{eq:max:big:and:E:no}
    \Pr\left(E \cap \left( \bigcup_{i \in [r]} \left\{ \big\langle \sigma_i(U),\sigma_i(U') \big\rangle \ge \lambda \right\}\right)\right) < \beta r e^{-C\sqrt{\lambda + 1}}.
  \end{align}

  Observe that, since \(x \le 2 + \cosh\sqrt{x}\) (\autoref{lem:le-coshsqrt}) and using \autoref{lem:moments} and linearity of expectation,
  \[
    \Ex\big[ f\big( Z_1,\ldots,Z_r \big) \big] \ge 0
  \]

  We hence have
  %
  \begin{align*}
    0 &\le \Ex\big[ f\big( Z_1,\ldots,Z_r \big) \big]\\
    &= \Ex\big[ f\big( Z_1,\ldots,Z_r \big)  \mathbf{1}_E \big] + \Ex\big[ f\big( Z_1,\ldots,Z_r \big) \mathbf{1}_{E^c} \big]\\
    &\le \Ex\left[ 3^r r \exp\bigg( \displaystyle\sum_{i = 1}^r \sqrt{ Z_i + 3r } \bigg)  \mathbf{1}_E \right] + \Ex\big[-1 \cdot \mathbf{1}_{E^c} \big]
  \end{align*}
  %
  where we use \autoref{lem:special:function:1} to bound the case for \(E\), and \autoref{lem:special:function:2} for \(E^c\).
  It follows that
  %
  \begin{equation}\label{eq:eventE:inequality}
    1 - \Pr(E) \le 3^r r \cdot \Ex\bigg[ \exp\bigg( \sum_{i = 1}^r \sqrt{ Z_i + 3r } \bigg) \mathbf{1}_E \bigg].
  \end{equation}

  Let \(M = \max \big\{ \big\langle \sigma_i(U),\sigma_i(U') \big\rangle : i \in [r] \big\}\). For any constant \(c \le C - 1\), we have
  %
  \begin{align}
  \exp(c\sqrt{M+1})\id_E&= \id_E \Bigl(e^{c\sqrt{1-1}}+ \int_{-1}^M \frac{ce^{c\sqrt{\lambda+1}}}{2\sqrt{\lambda+1}} d\lambda\Bigr)\\
  &= \id_E+ \int_{-1}^{\infty} \id_{E,\{M\geq \lambda\}}\frac{ce^{c\sqrt{\lambda+1}}}{2\sqrt{\lambda+1}} d\lambda.
  \end{align}
  %
  Given any \(c \leq C-1\) we take the expected value of both sides of the previous equation and apply Tonelli's theorem to obtain:
  %
  \begin{align*}
    \Ex\Big[ \exp\big( c \sqrt{M + 1} \big) \mathbf{1}_E \Big]
    & % = \int_{E} \exp\big( c \sqrt{M + 1}  \big) d\Pr
    \\
    & \le \, \Pr(E) + \int_{-1}^\infty \Pr\Big( \big\{ M \ge \lambda \big\} \cap E \Big) \cdot \frac{c}{2\sqrt{\lambda + 1}} \cdot e^{c \sqrt{\lambda + 1}} \,\mathrm{d}\lambda\\ %TODO not clear!
    & \le \, \beta +  \int_{-1}^\infty \frac{\beta r c}{2\sqrt{\lambda + 1}} \cdot e^{- C\sqrt{\lambda + 1}+c\sqrt{\lambda+1}} \,\mathrm{d}\lambda\\
    & \le \, \beta + \beta r \int_{-1}^\infty \frac{c}{2\sqrt{\lambda + 1}} \cdot e^{- \sqrt{\lambda + 1}} \,\mathrm{d}\lambda\\
    & \le \, \beta (cr + 1),
  \end{align*}
  %
  where in the second inequality we used (\autoref{eq:max:big:and:E:no}) and \(\Pr(E) \le \beta\); in the third we used \(c \le C - 1\); and in the last inequality we used the fact that \(\int_0^\infty \frac{1}{2\sqrt{x}} e^{-\sqrt{x}} \, \mathrm{d}x = 1\).
  Note that the second and third inequalities rely on the non-negativity of the integrand.
  In particular, applying this with \(c = \sqrt{3}r^{3/2}\), and recalling that \(C = 4r^{3/2} \ge c + 1\) and \(\beta = 3^{-4r}\), it follows that
  %
  \begin{align}
    1 - \Pr(E) &\le 3^r r \cdot \Ex\bigg[ \exp\bigg( \sum_{i = 1}^r \sqrt{ Z_i + 3r } \bigg) \mathbf{1}_E \bigg]\\
    &\le  3^r r \cdot \Ex\bigg[ \exp\bigg( r \cdot \max_{i \in [r]} \sqrt{ Z_i + 3r } \bigg) \mathbf{1}_E \bigg]\\
    &= 3^r r \cdot \Ex\bigg[ \exp\bigg( r \cdot \sqrt{ 3r} \cdot \sqrt{ (M + 1) } \bigg) \mathbf{1}_E \bigg]\\
    &\le 3^r r \cdot \beta (r^{2}\sqrt{3r} + 1) \le 1/3,
  \end{align}
  %
  which contradicts our assumption that \(\Pr(E) \le \beta\).

  Hence, there exists \(\lambda \ge -1\) such that
  %
  \begin{align}\label{eq:max:big:and:E}
    \beta r e^{-C\sqrt{\lambda + 1}} &\le \Pr\left(E \cap \left( \bigcup_{i \in [r]} \left\{ \big\langle \sigma_i(U),\sigma_i(U') \big\rangle \ge \lambda \right\}\right)   \right) \\
    &\le \sum_{i \in [r]} \Pr\left(E \cap \left\{ \big\langle \sigma_i(U),\sigma_i(U') \big\rangle \ge \lambda \right\} \right)\\
    &\le r \cdot \max_{i \in [r]} \Pr\left(E \cap \left\{ \big\langle \sigma_i(U),\sigma_i(U') \big\rangle \ge \lambda \right\} \right) ,
  \end{align}
  %
  and therefore there exists an \(i \in [r]\) as required.
\end{proof}