% The Key Lemma section - statement and algorithm overview
% This section states the key lemma and explains the select-and-boost algorithm

\section{The Key Lemma}

% ============================================================================
% KEY LEMMA STATEMENT
% ============================================================================

% [LEMMA] Key lemma: select and boost with geometric improvement
% Source: blueprint/paper/sections/section_select_and_boost_strategy.tex line 7-16 (\begin{lemma}\label{key:lemma})
% Deprecated blueprint: deprecated/04_The_Book_Theorem.tex line 4-20 (\begin{lemma}[Key lemma])
% Lean status: ✓ Partially implemented as `key` in KeyLemma.lean
% Label: lem:key-lemma (blueprint) / key:lemma (source)
% Dependencies: def:p, lem:geometric (from section 03), def:algorithm-constants
% TODO: Add key lemma statement with input: χ r-coloring, sets X,Y₁,...,Yᵣ, parameters α₁,...,αᵣ > 0
% TODO: Add existence of vertex x ∈ X, colour ℓ ∈ [r], sets X' ⊂ X, Y'ᵢ ⊂ Nᵢ(x) ∩ Yᵢ, λ ≥ -1
% TODO: Add bounds |X'| ≥ β e^{-C√(λ+1)} |X| and p_ℓ(X',Y'_ℓ) ≥ p_ℓ(X,Y_ℓ) + λα_ℓ
% TODO: Add conditions |Y'_i| = p_i(X,Y_i)|Y_i| and p_i(X',Y'_i) ≥ p_i(X,Y_i) - α_i

\begin{lemma}[Key lemma]
  \label{lem:key-lemma} % Source paper uses label key:lemma
  \paperref{Lemma 2.2}
  \uses{def:p, lem:geometric, def:algorithm-constants}
  \lean{key}
  %
  Let $r, n\in\N$. Set $\beta = 3^{-4r}$ and $C = 4r^{3/2}$.
  Let\/ $\chi$ be an\/ $r$-colouring of\/ $E(K_n)$, let\/ $X,Y_1,\ldots,Y_r \subset V(K_n)$ be non-empty sets of vertices, and let $\alpha_1,\ldots,\alpha_r > 0$. There exists a vertex $x \in X$, a colour $\ell \in [r]$, sets $X' \subset X$ and\/ $Y'_1,\ldots,Y'_r\,$ with\/ $Y'_i \subset N_i(x) \cap Y_i\,$ for each $i \in [r]$, and\/ $\lambda \ge -1$, such that
  %
  \begin{equation}\label{eq:key:ell}
    \beta e^{- C \sqrt{\lambda + 1}} |X| \le |X'| \qquad \text{and} \qquad p_\ell(X,Y_\ell) + \lambda \alpha_\ell \le p_\ell( X', Y'_\ell ) ,
  \end{equation}
  %
  and moreover
  %
  \begin{equation}\label{eq:key:alli}
    |Y'_i| = p_i(X,Y_i) |Y_i| \qquad \text{and} \qquad  p_i(X,Y_i) - \alpha_i \le p_i( X', Y'_i )
  \end{equation}
  %
  for every $i \in [r]$.
\end{lemma}
%
\begin{proof}
  \uses{def:p, lem:geometric}

  For each colour $i \in [r]$, define a function $\sigma_i \colon X \to \R^{\cup_i Y_i}$ as follows: for each $x \in X$, choose a set $N'_i(x) \subset N_i(x) \cap Y_i$ of size exactly $p_i|Y_i|$, where $p_i = p_i(X,Y_i)$, and set
  %
  $$\sigma_i(x) = \frac{\id_{N'_i(x)} - p_i\id_{Y_i}}{\sqrt{\alpha_ip_i|Y_i|}},$$
  %
  where $\id_S \in \{0,1\}^{\cup_i Y_i}$ denotes the indicator function of the set $S$. Note that, for any $x,y\in X$,
  %TODO why? mp direction suffices
  %
  $$\lambda \le \big\langle \sigma_i(x),\sigma_i(y) \big\rangle \quad \Leftrightarrow \quad \big( p_i + \lambda\alpha_i \big) p_i |Y_i| \le |N'_i(x) \cap N'_i(y)|.$$
  %
  Indeed, for every $x,y\in X$, we have 
  %
  \begin{multline}
    \begin{aligned}
            \alpha_ip_i|Y_i| \langle \sigma_i(x),\sigma_i(y) \rangle &= \langle \id_{N'_i(x)}- p_i\id_{Y_i},\id_{N'_i(y)}- p_i\id_{Y_i}\rangle\\
            &= \Bigl(\big\langle \id_{N'_i(x)},\id_{N'_i(y)}\big\rangle +p_i^2\langle \id_{Y_i},
            \id_{Y_i}\rangle - p_i \langle \id_{N'_i(x)},\id_{Y_i}\rangle- p_i \langle \id_{N'_i(y)},\id_{Y_i}\rangle\Bigr)\\
            &= \Bigl(|N'_i(x) \cap N'_i(y)|-p_i^2|Y_i| \Bigr),
    \end{aligned}
  \end{multline}
  %
  Where the last ineqaulity follows since $|N'_i(x)|= N'_i(y)|= p_i|Y_i|$ and both sets are subsets of $Y_i$.
  Now, by Lemma~\ref{lem:geometric}, there exists $\lambda \ge -1$ and colour $\ell \in [r]$ such that
  %
  \begin{equation}
    \label{eq: geometric-app}
    \beta e^{- C\sqrt{\lambda + 1}} \le \Pr\Big( \lambda  \le \big\langle \sigma_\ell(U),\sigma_\ell(U') \big\rangle \, \text{ and } \, -1 \le \big\langle \sigma_i(U), \sigma_i(U') \big\rangle \, \text{ for all } \, i \ne \ell \Big) .
  \end{equation}
  %
  where $U$, $U'$ are independent random variables distributed uniformly in the set~$X$. We claim that there exists a vertex $x \in X$ and a set $X' \subset X$
   %TODO
  such that,
  %
  $$\beta e^{- C \sqrt{\lambda + 1}} |X| \le |X'|$$ %TODO why?
  %
  and
  %
  $$\big( p_\ell + \lambda\alpha_\ell \big) p_\ell |Y_\ell | \le |N'_\ell(x) \cap N'_\ell(y)|$$
  %
  for every $y \in X'$, and
  %
  $$\big( p_i - \alpha_i \big) p_i |Y_i| \le |N'_i(x) \cap N'_i(y)|$$
  %
  for every $y \in X'$ and $i \in [r]$. To see this, we let 
  %
  $$ P= \Biggl\{(x,x'): x,x' \in X , \lambda  \le \big\langle \sigma_\ell(x),\sigma_\ell(x') \big\rangle \, \text{ and } \, -1 \le \big\langle \sigma_i(x), \sigma_i(x') \big\rangle \, \text{ for all } \, i \ne \ell  \Biggr\}.$$
  %
  Since $U$ and $U'$ are independent and uniformly distributed over $x$, Equation \ref{eq: geometric-app} implies that
  %
  \begin{equation}
    |P|\geq \beta e^{-C\sqrt{\lambda + 1}}|X|^2.
  \end{equation} 
  %
  However, by averaging, there must then exist a vertex $x \in X$ such that:
  %
  \begin{equation}
    |X'|:=\bigl|\{y: (x,y) \in P  \} \bigr| \ge \frac{\beta e^{-C\sqrt{\lambda+1}}|X|^2}{|X|}\geq  \beta e^{-C\sqrt{\lambda+1}}|X|.
  \end{equation}
  %
  We can them simply pick this $X'$ to be the desired subset.
  Setting $Y'_i = N'_i(x)$ for each $i \in [r]$, it follows that
  %
  \begin{multline}
      \begin{aligned}
      p_\ell(X,Y_\ell) + \lambda \alpha_\ell &= \frac{ \big( p_\ell + \lambda\alpha_\ell \big) p_\ell |Y_\ell |}{p_\ell |Y_\ell|}\\
      &= \frac{ \big( p_\ell + \lambda\alpha_\ell \big) p_\ell |Y_\ell |}{| N'_\ell (x)|}\\
      &= \min\bigg\{ \frac{ \big( p_\ell + \lambda\alpha_\ell \big) p_\ell |Y_\ell |}{| N'_\ell (x)|} : x' \in X' \bigg\}\\
      &\le \min\bigg\{ \frac{|N'_\ell(x') \cap  N'_\ell (x)|}{| N'_\ell (x)|} : x' \in X' \bigg\}\\
      &= \min\bigg\{ \frac{|N_\ell(x') \cap  N'_\ell (x)|}{| N'_\ell (x)|} : x' \in X' \bigg\}\\
      &= p_\ell\big( X', N'_\ell (x) \big) = p_\ell\big( X', Y'_\ell \big) \\
    \end{aligned}
  \end{multline}
  %
  and $$  p_i(X,Y_i) - \alpha_i \le \qquad p_i\big( X', Y'_i \big) $$
  for every $i \in [r]$, as required.
\end{proof}