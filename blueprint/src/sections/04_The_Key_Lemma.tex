% The Key Lemma section - statement and algorithm overview
% This section states the key lemma and explains the select-and-boost algorithm

\section{The Key Lemma}

% ============================================================================
% HELPER LEMMA: INNER PRODUCT EQUIVALENCE
% ============================================================================

We first establish a technical lemma that connects inner products to intersection sizes.
This corresponds to the \texttt{Λiff} statement in the Lean code.

\begin{lemma}[Inner product to intersection equivalence]
  \label{lem:inner-product-equivalence}
  \paperref{Lemma 2.2, implicit}
  \lean{Λiff}
  \leanok
  %
  Let \(X, Y \subset V(K_n)\) be non-empty sets of vertices, let \(\alpha, p > 0\), and let \(N'(x) \subset Y\) be sets of size exactly \(p|Y|\) for each \(x \in X\).
  Define
  %
  \begin{equation*}
    \sigma(x) = \frac{\id_{N'(x)} - p\id_{Y}}{\sqrt{\alpha p|Y|}}.
  \end{equation*}
  %
  Then for any \(x,y\in X\) and \(\lambda \in \R\),
  %
  \begin{equation*}
    \lambda \le \big\langle \sigma(x),\sigma(y) \big\rangle \quad \Leftrightarrow \quad \big( p + \lambda\alpha \big) p |Y| \le |N'(x) \cap N'(y)|.
  \end{equation*}
  %
\end{lemma}
%
\begin{proof}
  For every \(x,y\in X\), expanding the inner product gives
  %
  \begin{multline}
    \begin{aligned}
            \alpha p|Y| \langle \sigma(x),\sigma(y) \rangle &= \langle \id_{N'(x)}- p\id_{Y},\id_{N'(y)}- p\id_{Y}\rangle\\
            &= \Bigl(\big\langle \id_{N'(x)},\id_{N'(y)}\big\rangle +p^2\langle \id_{Y},
            \id_{Y}\rangle - p \langle \id_{N'(x)},\id_{Y}\rangle- p \langle \id_{N'(y)},\id_{Y}\rangle\Bigr)\\
            &= \Bigl(|N'(x) \cap N'(y)|-p^2|Y| \Bigr),
    \end{aligned}
  \end{multline}
  %
  where the last equality follows since \(|N'(x)|= |N'(y)|= p|Y|\) and both sets are subsets of \(Y\).
  %
  Thus, for any \(\lambda \in \R\), we have
  %
  \begin{align*}
    \lambda \le \langle \sigma(x),\sigma(y) \rangle
    &\iff \lambda \alpha p|Y| \le \alpha p|Y| \langle \sigma(x),\sigma(y) \rangle \\
    &\iff \lambda \alpha p|Y| \le |N'(x) \cap N'(y)|-p^2|Y| \\
    &\iff p^2|Y| + \lambda \alpha p|Y| \le |N'(x) \cap N'(y)| \\
    &\iff \big( p + \lambda\alpha \big) p |Y| \le |N'(x) \cap N'(y)|,
  \end{align*}
  %
  where the first equivalence uses the positivity \(\alpha p|Y| > 0\), and the last follows by factoring.
\end{proof}

% ============================================================================
% KEY LEMMA STATEMENT
% ============================================================================

We can deduce \autoref{key:lemma} from \autoref{lem:lambda}.

\begin{lemma}[Key lemma]
  \label{lem:key-lemma} % Source paper uses label key:lemma
  \paperref{Lemma 2.2}
  \uses{def:p, lem:geometric, lem:inner-product-equivalence}
  \lean{key}
  %
  Let \(r, n\in\N\). Set \(\beta = 3^{-4r}\) and \(C = 4r^{3/2}\).
  Let \(\chi\) be an \(r\)-colouring of \(E(K_n)\), let \(X,Y_1,\ldots,Y_r \subset V(K_n)\) be non-empty sets of vertices, and let \(\alpha_1,\ldots,\alpha_r > 0\). There exists a vertex \(x \in X\), a colour \(\ell \in [r]\), sets \(X' \subset X\) and \(Y'_1,\ldots,Y'_r\,\) with \(Y'_i \subset N_i(x) \cap Y_i\,\) for each \(i \in [r]\), and \(\lambda \ge -1\), such that
  %
  \begin{equation}\label{eq:key:ell}
    \beta e^{- C \sqrt{\lambda + 1}} |X| \le |X'| \qquad \text{and} \qquad p_\ell(X,Y_\ell) + \lambda \alpha_\ell \le p_\ell( X', Y'_\ell ) ,
  \end{equation}
  %
  and moreover
  %
  \begin{equation}\label{eq:key:alli}
    |Y'_i| = p_i(X,Y_i) |Y_i| \qquad \text{and} \qquad  p_i(X,Y_i) - \alpha_i \le p_i( X', Y'_i )
  \end{equation}
  %
  for every \(i \in [r]\).
\end{lemma}
%
\begin{proof}
  \uses{def:p, lem:geometric, lem:inner-product-equivalence}

  % ========================================================================
  % STEP 1: Construct N'_i(x) and define σ_i
  % Lean: lines 92-109 (defines N', proves N'sub, N'subN, N'card, defines σ)
  % ========================================================================

  For each colour \(i \in [r]\), we will define a function \(\sigma_i \colon X \to \R^{\cup_i Y_i}\).
  %
  First, for each \(x \in X\), we choose a set \(N'_i(x) \subset N_i(x) \cap Y_i\) of size exactly \(p_i|Y_i|\), where \(p_i = p_i(X,Y_i)\).
  %
  \begin{remark}
    Such a choice is always possible: by definition of \(p_i(X,Y_i)\), we have \(p_i|Y_i| \le \min_{x \in X} |N_i(x) \cap Y_i|\), so every vertex \(x \in X\) has at least \(p_i|Y_i|\) neighbors of color \(i\) in \(Y_i\).
    In Lean (lines 92-95), this is proven formally via \texttt{p'} and \texttt{exists\_subset\_card\_eq}.
  \end{remark}
  %
  We record the following properties of \(N'_i(x)\):
  %
  \begin{itemize}
    \item \(N'_i(x) \subset N_i(x) \cap Y_i\) for each \(x \in X\) and \(i \in [r]\).
    \item \(|N'_i(x)| = p_i|Y_i|\) for each \(x \in X\) and \(i \in [r]\).
  \end{itemize}
  %
  Now define
  %
  \begin{equation}\label{eq:sigma-def}
    \sigma_i(x) = \frac{\id_{N'_i(x)} - p_i\id_{Y_i}}{\sqrt{\alpha_ip_i|Y_i|}},
  \end{equation}
  %
  where \(\id_S \in \{0,1\}^{\cup_i Y_i}\) denotes the indicator function of the set \(S\).

  % ========================================================================
  % STEP 2: Apply the Geometric Lemma
  % Lean: lines 117-122 (setup measure space, call geometric, destructure)
  % ========================================================================

  By \autoref{lem:geometric}, applied with these functions \(\sigma_i\) and with \(U\), \(U'\) independent random variables distributed uniformly in the set~\(X\), there exists \(\lambda \ge -1\) and colour \(\ell \in [r]\) such that
  %
  \begin{equation}
    \label{eq:geometric-app}
    \beta e^{- C\sqrt{\lambda + 1}} \le \Pr\Big( \lambda  \le \big\langle \sigma_\ell(U),\sigma_\ell(U') \big\rangle \, \text{ and } \, -1 \le \big\langle \sigma_i(U), \sigma_i(U') \big\rangle \, \text{ for all } \, i \ne \ell \Big) .
  \end{equation}
  %

  % ========================================================================
  % STEP 3: Derandomization via averaging (pigeonhole principle)
  % Lean: lines 124-139 (calls pidgeon_thing, proves X' nonempty)
  % ========================================================================

  We now convert this probabilistic statement into a deterministic existence statement via an averaging argument.
  (In Lean, this corresponds to the \texttt{pidgeon\_thing} lemma at lines 124-139.)

  Define the set of ``good pairs''
  %
  \begin{equation}\label{eq:P-def}
     P= \Biggl\{(x,x'): x,x' \in X , \lambda  \le \big\langle \sigma_\ell(x),\sigma_\ell(x') \big\rangle \, \text{ and } \, -1 \le \big\langle \sigma_i(x), \sigma_i(x') \big\rangle \, \text{ for all } \, i \ne \ell  \Biggr\}.
  \end{equation}
  %
  Since \(U\) and \(U'\) are independent and uniformly distributed over \(X\), the probability in \eqref{eq:geometric-app} equals \(|P|/|X|^2\).
  Therefore,
  %
  \begin{equation}\label{eq:P-size}
    |P|\geq \beta e^{-C\sqrt{\lambda + 1}}|X|^2.
  \end{equation}
  %
  By the pigeonhole principle (averaging argument), there must exist a vertex \(x \in X\) such that the number of partners \(y\) satisfying \((x,y) \in P\) is at least the average:
  %
  \begin{equation}\label{eq:Xprime-def}
    |X'|:=\bigl|\{y: (x,y) \in P  \} \bigr| \ge \frac{|P|}{|X|} \geq  \beta e^{-C\sqrt{\lambda+1}}|X|.
  \end{equation}
  %
  Fix such a vertex \(x \in X\) and define \(X' = \{y \in X : (x,y) \in P\}\).

  % ========================================================================
  % STEP 4: Define Y'_i and verify the inequalities
  % Lean: lines 141-201 (defines Y', proves Y'card, proves Y'lee with factor function)
  % ========================================================================

  Set \(Y'_i = N'_i(x)\) for each \(i \in [r]\).
  Note that \(|Y'_i| = |N'_i(x)| = p_i|Y_i|\) by construction, which gives the first part of \eqref{eq:key:alli}.

  We now verify the density inequalities.
  By the definition of \(X'\) and \(P\), for every \(y \in X'\), we have:
  %
  \begin{align}
    \lambda &\le \big\langle \sigma_\ell(x),\sigma_\ell(y) \big\rangle, \label{eq:lambda-ell}\\
    -1 &\le \big\langle \sigma_i(x), \sigma_i(y) \big\rangle \quad \text{for all } i \ne \ell. \label{eq:minus1-i}
  \end{align}
  %
  Applying \autoref{lem:inner-product-equivalence} (the forward direction), we obtain:
  %
  \begin{align}
    \big( p_\ell + \lambda\alpha_\ell \big) p_\ell |Y_\ell | &\le |N'_\ell(x) \cap N'_\ell(y)|, \label{eq:bound-ell}\\
    \big( p_i - \alpha_i \big) p_i |Y_i| &\le |N'_i(x) \cap N'_i(y)| \quad \text{for all } i \ne \ell. \label{eq:bound-i}
  \end{align}
  %

  \textbf{Case 1: The selected color \(\ell\).}
  %
  We have
  %
  \begin{align*}
      p_\ell(X,Y_\ell) + \lambda \alpha_\ell &= \frac{ \big( p_\ell + \lambda\alpha_\ell \big) p_\ell |Y_\ell |}{p_\ell |Y_\ell|}\\
      &= \frac{ \big( p_\ell + \lambda\alpha_\ell \big) p_\ell |Y_\ell |}{| N'_\ell (x)|} \qquad \text{(since \(|N'_\ell(x)| = p_\ell|Y_\ell|\))}\\
      &= \min\bigg\{ \frac{ \big( p_\ell + \lambda\alpha_\ell \big) p_\ell |Y_\ell |}{| N'_\ell (x)|} : y \in X' \bigg\} \qquad \text{(constant function)}\\
      &\le \min\bigg\{ \frac{|N'_\ell(x) \cap  N'_\ell (y)|}{| N'_\ell (x)|} : y \in X' \bigg\} \qquad \text{(by \eqref{eq:bound-ell})}\\
      &\le \min\bigg\{ \frac{|N_\ell(y) \cap  N'_\ell (x)|}{| N'_\ell (x)|} : y \in X' \bigg\} \qquad \text{(since \(N'_\ell(y) \subset N_\ell(y)\))}\\
      &= p_\ell\big( X', N'_\ell (x) \big) = p_\ell\big( X', Y'_\ell \big), \qquad \text{(definition of \(p_\ell\))}
  \end{align*}
  %
  which establishes the second part of \eqref{eq:key:ell}.

  \textbf{Case 2: The other colors \(i \ne \ell\).}
  %
  The calculation is identical, replacing \(\lambda\) with \(-1\):
  %
  \begin{align*}
      p_i(X,Y_i) - \alpha_i &= \frac{ \big( p_i - \alpha_i \big) p_i |Y_i |}{p_i |Y_i|}\\
      &= \frac{ \big( p_i - \alpha_i \big) p_i |Y_i |}{| N'_i (x)|} \qquad \text{(since \(|N'_i(x)| = p_i|Y_i|\))}\\
      &= \min\bigg\{ \frac{ \big( p_i - \alpha_i \big) p_i |Y_i |}{| N'_i (x)|} : y \in X' \bigg\} \qquad \text{(constant function)}\\
      &\le \min\bigg\{ \frac{|N'_i(x) \cap  N'_i (y)|}{| N'_i (x)|} : y \in X' \bigg\} \qquad \text{(by \eqref{eq:bound-i})}\\
      &\le \min\bigg\{ \frac{|N_i(y) \cap  N'_i (x)|}{| N'_i (x)|} : y \in X' \bigg\} \qquad \text{(since \(N'_i(y) \subset N_i(y)\))}\\
      &= p_i\big( X', N'_i (x) \big) = p_i\big( X', Y'_i \big), \qquad \text{(definition of \(p_i\))}
  \end{align*}
  %
  which establishes the second part of \eqref{eq:key:alli} for all \(i \ne \ell\).

  \begin{remark}[Lean structure]
    In the Lean proof (lines 152-201), both cases are handled simultaneously using a \texttt{factor} function defined as \(\text{factor}(i) = \lambda\) if \(i = \ell\) and \(-1\) otherwise.
    This allows a unified treatment via the auxiliary functions \texttt{f} and \texttt{g}.
  \end{remark}
\end{proof}