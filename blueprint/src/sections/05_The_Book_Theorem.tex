% The Book Theorem section - statement and proof using key lemma
% This section states and proves the book theorem using the select-and-boost algorithm

\section{The Book Theorem}

% ============================================================================
% THE ALGORITHM
% ============================================================================

\begin{algorithm}[The Multicolour Book Algorithm in Balister et al.]\label{alg:book}
  %Let \(\chi\) be an \(r\)-colouring of \(E(K_n)\), let \(X\) and \(Y_1,\ldots, Y_r\) be disjoint sets of vertices of \(K_n\), and 
  Set \(T_1 = \cdots = T_r = \emptyset\), and repeat the following steps until either \(X = \emptyset\) or \(\max\big\{ |T_i| : i \in [r] \big\} = t\). 
  \begin{enumerate}
  \item\label{Alg:Step1} Applying the key lemma: let the vertex \(x \in X\), the colour \(\ell \in [r]\), the sets \(X' \subset X\) and \(Y'_1,\ldots,Y'_r\), and \(\lambda \ge -1\) be given by Lemma~\ref{lem:key-lemma}, applied with
  \begin{equation}\label{def:alpha}
  \alpha_i = \frac{p_i(X,Y_i) - p_0 + \delta}{t}
  \end{equation}
  for each \(i \in [r]\), and go to Step~2.\smallskip
  \item\label{Alg:Step2} Colour step: If \(\lambda \le \lambda_0\), then choose a colour \(j \in [r]\) such that the set
  %
  \begin{equation*}
    X'' = N_j(x) \cap X'
  \end{equation*}
  % 
  has at least \((|X'| - 1)/r\) elements, and update the sets as follows:
  %
  \begin{equation*}
    X \to X'', \qquad Y_j \to Y'_j \qquad \text{and} \qquad T_j \to T_j \cup \{x\}
  \end{equation*}
  %
  and go to Step~1. Otherwise go to Step~3.\smallskip
  \item\label{Alg:Step3} Density-boost step: If \(\lambda > \lambda_0\), then we update the sets as follows:
  %
  \begin{equation*}
    X \to X' \qquad \text{and} \qquad Y_\ell \to Y'_\ell,
  \end{equation*}
  %
  and go to Step~1.
  \end{enumerate}  
\end{algorithm}

In this section we will use the multicolour book algorithm to prove Theorem~\ref{thm:book}. To do so, we will first prove a few simple lemmas about the sets produced by the algorithm when applied with arbitrary inputs, and then apply these bounds to our setting. Recall that we are given an \(r\)-colouring \(\chi\) of \(E(K_n)\) and
sets \(X,Y_1,\ldots,Y_r \subset V(K_n)\), define
%
\begin{equation}
	\label{def:p0}
	p_0 = \min\big\{ p_i(X,Y_i) : i \in [r] \big\},
\end{equation}
%
and run the algorithm for some \(t \in \N\), \(\lambda_0 \ge -1\) and \(\delta > 0\).

In order to simplify the statements of the lemmas below, let us write \(X(s)\) and \(Y_i(s)\) for the sets \(X\) and \(Y_i\) after \(s\) steps of the algorithm, and set
%
\begin{equation*}
	p_i(s) = p_i\big( X(s), Y_i(s) \big) \qquad \text{and} \qquad \alpha_i(s) = \frac{p_i(s) - p_0 + \delta}{t}.
\end{equation*}
%
Let us also write \(\ell(s) \in [r]\) and \(\lambda(s) \ge -1\) for the colour and number given by the application of Lemma~\ref{key:lemma} to the sets \(X(s)\) and \(Y_1(s),\dots,Y_r(s)\), and define
%
\begin{equation*}
	\cB_i(s) = \big\{ 0 \le j < s \,:\, \ell(j) = i \, \text{ and } \, \lambda(j) > \lambda_0 \big\},
\end{equation*}
%
for each \(i \in [r]\) and \(s \in \N\), to be the set of density-boost steps in colour \(i\) during the first \(s\) steps of the algorithm. We begin by noting the following lower bound on \(p_i(s)\).

% ============================================================================
% AUXILIARY LEMMAS FOR BOOK THEOREM PROOF
% ============================================================================

\begin{lemma}[Lower bound for \(p_i\) during algorithm]
  \label{lem:pi:lower:bound} % Source paper uses same label
  \paperref{Lemma 4.1}
  \uses{alg:book, lem:key-lemma}
  %
  For each \(i \in [r]\) and \(s \in \N\), 
  %
  \begin{equation}\label{eq:pi:lower:bound}
    p_i(s) - p_0 + \delta \, \ge \, \delta \cdot \bigg( 1 - \frac{1}{t} \bigg)^{t} \prod_{j \in \cB_i(s)} \bigg( 1 + \frac{\lambda(j)}{t} \bigg).
  \end{equation}
  %
\end{lemma}

\begin{proof}
  Note first that if \(Y_i(s+1) = Y_i(s)\), then \(p_i(s+1) \ge p_i(s)\), since the minimum degree does not decrease when we take a subset of \(X(s)\). When we perform a colour step in colour \(i\), % (that is, we add \(x\) to \(T_i\)), 
  we have \(p_i(s+1) \ge p_i(s) - \alpha_i(s)\), by Lemma~\ref{lem:key-lemma}, and hence
  %
  \begin{equation*}
    p_i(s+1) - p_0 + \delta \ge \bigg( 1 - \frac{1}{t} \bigg) \big( p_i(s) - p_0 + \delta \big),
  \end{equation*}
  %
  by our choice of \(\alpha_i(s)\). Similarly, when we perform a density-boost step in colour \(i\)  
  we have \(p_i(s+1) \ge p_i(s) + \lambda(s) \alpha_i(s)\), by Lemma~\ref{lem:key-lemma}, and hence
  %
  \begin{equation*}
    p_i(s+1) - p_0 + \delta \ge \bigg( 1 + \frac{\lambda(s)}{t} \bigg) \big( p_i(s) - p_0 + \delta \big).
  \end{equation*}
  %
  Recalling that there are at most \(t\) colour steps in colour \(i\), and that \(p_i(0) \ge p_0\), by the definition~\eqref{def:p0} of \(p_0\), the claimed bound follows. 
\end{proof}

Before continuing, let's note a couple of important consequences of Lemma~\ref{lem:pi:lower:bound}. First, it implies that neither \(p_i(s)\) nor \(\alpha_i(s)\) can get too small. 

\begin{lemma}[Minimum bounds on \(p_i\) and \(\alpha_i\)]
  \label{lem:pi:min} % Source paper uses same label
  \paperref{Lemma 4.2}
  \uses{lem:pi:lower:bound, alg:book}
  %
  If \(t \ge 2\), then 
  %
  \begin{equation*}
    p_i(s) \, \ge \, p_0 - \frac{3\delta}{4} \qquad \text{and} \qquad \alpha_i(s) \, \ge \, \frac{\delta}{4t}
  \end{equation*}
  %
  for every \(i \in [r]\) and \(s \in \N\). 
\end{lemma}
%
\begin{proof}
  Both bounds follow immediately from~\eqref{eq:pi:lower:bound} and the definition of \(\alpha_i(s)\). 
\end{proof}

It also implies the following bound on the number of density-boost steps. 
%
\begin{lemma}[Upper bound on density-boost steps]
  \label{lem:Bi:max} % Source paper uses same label
  \paperref{Lemma 4.3}
  \uses{lem:pi:min}
  %
  If \(t \ge \lambda_0\) and \(\delta \le 1/4\), then
  %
  \begin{equation*}
    |\cB_i(s)| \le \frac{4 \log(1/\delta)}{\lambda_0} \cdot t
  \end{equation*}
  %
  for every \(i \in [r]\) and \(s \in \N\). 
\end{lemma}
%
\begin{proof}
    \uses{lem:pi:lower:bound}
  Since \(\lambda(j) > \lambda_0\) for every \(j \in \cB_i(s)\), and \(p_i(s) \le 1\), it follows from~\eqref{eq:pi:lower:bound} that
  %
  \begin{equation*}
    \frac{\delta}{4} \bigg( 1 + \frac{\lambda_0}{t} \bigg)^{|\cB_i(s)|} \le 1 + \delta.
  \end{equation*}
  %
  Since \(t \ge \lambda_0\) and \(\delta \le 1/4\), the claimed bound follows. 
\end{proof}

Lemmas~\ref{lem:pi:min} and~\ref{lem:Bi:max} together provide a lower bound on the size of the set \(Y_i(s)\). 

\begin{lemma}[Lower bound for \(Y_i\) set sizes]
  \label{lem:Y:lower:bound} % Source paper uses same label
  \paperref{Lemma 4.4}
  \uses{lem:pi:min, lem:Bi:max, alg:book}
  %
  If \(t \ge 2\), then
  %
  \begin{equation*}
    |Y_i(s)| \ge \bigg( p_0 - \frac{3\delta}{4} \bigg)^{t + |\cB_i(s)|} |Y_i(0)|
  \end{equation*}
  %
  for every \(i \in [r]\) and \(s \in \N\). 
\end{lemma}
%
\begin{proof}
  Note that \(Y_i(j+1) \ne Y_i(j)\) for at most \(t + |\cB_i(s)|\) of the first \(s\) steps, and for those steps we have
  %
  \begin{equation*}
    |Y_i(j+1)| = p_i(j) |Y_i(j)| \ge \bigg( p_0 - \frac{3\delta}{4} \bigg) |Y_i(j)|,
  \end{equation*}
  % 
  by~\eqref{eq:key:alli} and Lemma~\ref{lem:pi:min}.
\end{proof}

Finally, we need to bound the size of the set \(X(s)\). To do so, set \(\eps = (\beta / r) e^{- C \sqrt{\lambda_0 + 1}}\), and define \(\cB(s) = \cB_1(s) \cup \cdots \cup \cB_r(s)\) to be the set of all density-boost steps. 

\begin{lemma}[Lower bound for reservoir X size]
  \label{lem:X:lower:bound} % Source paper uses same label
  \paperref{Lemma 4.5}
  \uses{lem:sum:of:lambdas, alg:book}
  %
  For each \(s \in \N\), 
  %
  \begin{equation}\label{eq:X:lower:bound}
    |X(s)| \ge \eps^{rt + |\cB(s)|} \exp\bigg( - C \sum_{j \in \cB(s)} \sqrt{\lambda(j)+1}\,\, \bigg) |X(0)| - rt.
  \end{equation}
  %
\end{lemma}
%
\begin{proof}
  If \(\lambda(j) \le \lambda_0\), then by~\eqref{eq:key:ell} and Step~2 of the algorithm we have
  %
  \begin{equation*}
    |X(j+1)| \ge \frac{\beta e^{- C \sqrt{\lambda_0 + 1}}}{r} \cdot |X(j)| - 1 = \eps |X(j)| - 1.
  \end{equation*}
  % 
  On the other hand, if \(\lambda(j) > \lambda_0\), then \(j \in \cB(s)\), and we have 
  %
  \begin{equation*}
    |X(j+1)| \ge \beta e^{- C \sqrt{\lambda(j) + 1}} |X(j)|,
  \end{equation*}
  % 
  by~\eqref{eq:key:ell} and Step~3 of the algorithm. Since there are at most \(rt\) colour steps, and \(\beta \ge \eps\), the claimed bound follows.
\end{proof}

We will use the following lemma to bound the right-hand side of~\eqref{eq:X:lower:bound}.

\begin{lemma}[Sum of \(\lambda\) parameters bound]
  \label{lem:sum:of:lambdas} % Source paper uses same label
  \paperref{Lemma 4.6}
  \uses{lem:Bi:max, alg:book}
  %
  If \(t \ge \lambda_0 / \delta > 0\) and \(\delta \le 1/4\), then
  %
  \begin{equation*}
    \sum_{j \in \cB(s)} \sqrt{\lambda(j)} \le \frac{7r \log(1/\delta)}{\sqrt{\lambda_0}} \cdot t
  \end{equation*}
  %
  for every \(s \in \N\). 
\end{lemma}
%
\begin{proof}
  Observe first that, by Lemma~\ref{lem:pi:lower:bound}, we have 
  %
  \begin{equation}\label{eq:B:condition}
    \frac{\delta}{4} \prod_{j \in \cB_i(s)} \bigg( 1 + \frac{\lambda(j)}{t} \bigg) \le 1 + \delta
  \end{equation}
  %
  for each \(i \in [r]\). Note also that \(\lambda_0 \le \lambda(j) \le 5t/\delta\) for every \(j \in \cB(s)\), where the lower bound holds by the definition of \(\cB(s)\), and the upper bound by~\eqref{eq:B:condition} and since \(\delta \le 1/4\). Now, since \(\log(1+x) \ge \min\{ x/2,1\}\) for all \(x > 0\), it follows that
  %
  \begin{equation}\label{eq:not:convexity}
    \frac{ \sqrt{\lambda(j)} }{\log(1 + \lambda(j)/t) } \le \max\bigg\{ \frac{ 2t }{ \sqrt{\lambda_0} }, \, \sqrt{ \frac{5t}{\delta}} \bigg\} \le \frac{ 3t }{ \sqrt{\lambda_0} },
  \end{equation}
  %
  where in the second inequality we used our assumption that \(t \ge \lambda_0 / \delta\). It now follows that
  %
  \begin{equation*}
    \sum_{j \in \cB_i(s)} \sqrt{\lambda(j)} \le \frac{ 3t }{ \sqrt{\lambda_0} } \sum_{j \in \cB_i(s)} \log \bigg( 1 + \frac{\lambda(j)}{t} \bigg) \le \frac{7 \log(1/\delta)}{\sqrt{\lambda_0}} \cdot t,
  \end{equation*}
  %
  where the first inequality holds by~\eqref{eq:not:convexity}, and the second follows from~\eqref{eq:B:condition}, since \(\delta \le 1/4\). Summing over \(i \in [r]\), we obtain the claimed bound. 
\end{proof}

% ============================================================================
% BOOK THEOREM
% ============================================================================

\begin{theorem}[Book theorem]
  \label{thm:book} % Source paper uses same label
  \paperref{Theorem 2.1}
  \uses{def:book, def:edge-colouring, def:colour-neighborhood, alg:book, lem:pi:lower:bound, lem:pi:min, lem:Bi:max, lem:Y:lower:bound, lem:X:lower:bound, lem:sum:of:lambdas}
  %
  Let \(\chi\) be an \(r\)-colouring of \(E(K_n)\), and let \(X,Y_1,\ldots,Y_r \subset V(K_n)\). % be disjoint sets. 
  For every \(p > 0\) and \(\mu \ge 2^{10} r^3\), and every \(t,m \in \N\) with \(t \ge \mu^5 / p\), the following holds. If 
  %
  \begin{equation*}
    |N_i(x) \cap Y_i| \ge p|Y_i|
  \end{equation*}
  %
  for every \(x \in X\) and \(i \in [r]\), and moreover
  %
  \begin{equation*}
    |X| \ge \bigg( \frac{\mu^2}{p} \bigg)^{\mu r t} \qquad \text{and} \qquad |Y_i| \ge \bigg( \frac{e^{2^{13} r^3 / \mu^2}}{p} \bigg)^t \, m,
  \end{equation*}
  % 
  then \(\chi\) contains a monochromatic \((t,m)\)-book.
\end{theorem}
%
\begin{proof}
  Recall that we are given an \(r\)-colouring \(\chi\) of \(E(K_n)\) and a collection of %disjoint 
  sets \(X,Y_1,\ldots,Y_r \subset V(K_n)\) with
  %
  \begin{equation}\label{eq:book:thm:conditions}
    |X| \ge \bigg( \frac{\mu^2}{p} \bigg)^{\mu r t} \qquad \text{and} \qquad |Y_i| \ge \bigg( \frac{e^{2^{13} r^3 / \mu^2}}{p} \bigg)^t \, m
  \end{equation}
  %
  for each \(i \in [r]\), for some \(p > 0\), \(\mu \ge 2^{10} r^3\) and \(t,m \in \N\) with \(t \ge \mu^5 / p\), and moreover
  %
  \begin{equation}\label{eq:book:thm:min:degree}
    |N_i(x) \cap Y_i| \ge p|Y_i|
  \end{equation}
  %
  for every \(x \in X\) and \(i \in [r]\). 
  %Our task is to show that the multicolour book algorithm produces a monochromatic \((t,m)\)-book. 
  We will run the multicolour book algorithm with
  %
  \begin{equation*}
    \delta = \frac{p}{\mu^2} \qquad \text{and} \qquad \lambda_0 = \bigg( \frac{\mu \log(1/\delta)}{8C} \bigg)^2,
  \end{equation*}
  %
  where \(C = 4r^{3/2}\) (as before), and show that it ends with
  %
  \begin{equation*}
    \max \big\{ |T_i(s)| : i \in [r] \big\} = t \qquad \text{and} \qquad \min\big\{ |Y_i(s)| : i \in [r] \big\} \ge m,
  \end{equation*}
  %
  and therefore produces a monochromatic \((t,m)\)-book.
  
  To do so, we just need to bound the sizes of the sets \(X(s)\) and \(Y_i(s)\) from below. Observe that \(t \ge \lambda_0\) and \(\delta \le 1/4\), and therefore, by Lemma~\ref{lem:Bi:max}, that
  %
  \begin{equation}\label{eq:Bi:final:bound}
    |\cB_i(s)| \, \le \, \frac{4 \log(1/\delta)}{\lambda_0} \cdot t \, = \, \frac{2^{12} r^3}{\mu^2 \log(1/\delta)} \cdot t \, \le \, t
  \end{equation}
  %
  for every \(i \in [r]\) and \(s \in \N\). Since \(p_0 \ge p\), by~\eqref{eq:book:thm:min:degree} and the definition of \(p_0\), it follows that  
  %
  \begin{equation*}
    |Y_i(s)| \, \ge \, \bigg( p - \frac{3\delta}{4} \bigg)^{t + |\cB_i(s)|} |Y_i| \, \ge \, e^{-2\delta t / p} \, p^{|\cB_i(s)|} \, \big( e^{2^{13} r^3 / \mu^2} \big)^t \, m,
  \end{equation*}
  %
  where the first inequality holds by Lemma~\ref{lem:Y:lower:bound}, and the second by~\eqref{eq:book:thm:conditions} and~\eqref{eq:Bi:final:bound}. Noting that \(p^{|\cB_i(s)|} \ge e^{-2^{12} r^3 t / \mu^2}\), by~\eqref{eq:Bi:final:bound}, and that \(\delta / p = 1/\mu^2\), it follows that 
  %
  \begin{equation*}
    |Y_i(s)| \, \ge \, e^{-2 t / \mu^2} \big( e^{2^{12} r^3 / \mu^2} \big)^t \, m \, \ge \, m,
  \end{equation*}
  %
  as claimed. To bound \(|X(s)|\), recall~\eqref{eq:X:lower:bound} and observe that 
  %
  \begin{equation*}
    \eps^{rt + |\cB(s)|}  \ge \bigg( \frac{\beta}{r} \cdot e^{- C \sqrt{\lambda_0 + 1}} \bigg)^{2rt} \ge \big( e^{- 4C \sqrt{\lambda_0}} \big)^{rt} = \delta^{\mu r t/2} = \bigg( \frac{\mu^2}{p} \bigg)^{-\mu r t / 2},
  \end{equation*}
  %
  where the first step holds because \(\eps = (\beta / r) e^{- C \sqrt{\lambda_0 + 1}}\) and \(|\cB(s)| \le rt\), by~\eqref{eq:Bi:final:bound}, and the second step holds because \(\beta = 3^{-4r}\) and \(\lambda_0 \ge 2^{10} r^3\). Moreover, we have
  %
  \begin{equation*}
    \sum_{j \in \cB(s)} \sqrt{\lambda(j)+1} \le \frac{8r \log(1/\delta)}{\sqrt{\lambda_0}} \cdot t = \frac{2^6Crt}{\mu},
  \end{equation*}
  %
  by Lemma~\ref{lem:sum:of:lambdas} and our choice of \(\lambda_0\), and therefore 
  %
  \begin{equation*}
    |X(s)| \ge \bigg( \frac{\mu^2}{p} \bigg)^{\mu r t / 2} \exp\bigg( - \frac{2^6C^2rt}{\mu} \bigg) - rt > 0,
  \end{equation*}
  %
  for every \(s \in \N\), by Lemma~\ref{lem:X:lower:bound} and since \(\mu \ge 2^{10} r^3\) and \(C = 4r^{3/2}\). It follows that the algorithm produces a monochromatic \((t,m)\)-book, as claimed. 
\end{proof}