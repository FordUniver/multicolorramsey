% Deducing the Main Result section - Erdős-Szekeres steps and final assembly
% This section combines all results to prove the main theorem

\section{Deducing the Bound on the Ramsey Number}

% ============================================================================
% ERDŐS-SZEKERES AUXILIARY LEMMAS
% ============================================================================

% [LEMMA] Single Erdős-Szekeres step
% Source: blueprint/paper/sections/section_deducing_the_bound_on_rrk.tex line 13 (\begin{lemma}\label{lem:ESz:steps})
% Deprecated blueprint: deprecated/05_Deducing_the_Bound_on_the_Ramsey_Number.tex line 4-21 (\begin{lemma}[Lemma 5.2])
% Lean status: ✗ Not implemented
% Label: lem:ESz:steps
% Dependencies: def:edge-colouring, def:colour-neighborhood, def:book
% TODO: Add lemma for n,r ∈ ℕ and ε > 0 with r-colouring of E(K_n)
% TODO: Add existence of disjoint sets S₁,...,Sᵣ and W with bounds
% TODO: Add |W| ≥ ((1+ε)/r)^{∑|Sⱼ|} n and |N_i(w) ∩ W| ≥ (1/r - ε)|W| - 1
% TODO: Add (S_i,W) is monochromatic book in colour i for each i ∈ [r]

\begin{lemma}[Single Erdős-Szekeres step]
  \label{lem:ESz:steps} % Source paper uses same label
  \paperref{Lemma 5.2}
  \uses{def:edge-colouring, def:colour-neighborhood, def:book}
  %
  Given\/ $n,r \in \N$ and\/ $\eps > 0$, and an $r$-colouring of $E(K_n)$, the following holds. There exist disjoint sets of vertices\/ $S_1,\dots,S_r$ and\/ $W$ such that, for every $i \in [r]$, 
  $$|W| \ge \bigg( \frac{1+\eps}{r} \bigg)^{\sum_{j = 1}^r |S_j|} n \qquad \text{and} \qquad |N_i(w) \cap W| \ge \bigg( \frac{1}{r} - \eps \bigg) |W| - 1$$ 
  for every $w \in W$, and $(S_i,W)$ is a monochromatic book in colour~$i$.
\end{lemma}
%
\begin{proof}
  We use induction on~$n$. Note that if $n \le r$ or $r = 1$, then the sets $S_1 = \cdots = S_r = \emptyset$ and $W = V(K_n)$ have the required properties. We may therefore assume that $n > r \ge 2$, and that there exists a vertex $x \in V(K_n)$ and a colour $\ell \in [r]$ such that
  $$|N_\ell(x)| < \bigg( \frac{1}{r} - \eps \bigg) n - 1,$$
  and hence there exists a different colour $j \ne \ell$ such that 
  $$|N_j(x)| \ge \frac{1}{r-1} \bigg( n - \bigg( \frac{1}{r} - \eps \bigg) n \bigg) \ge \bigg( \frac{1 + \eps}{r} \bigg) n.$$
  Applying the induction hypothesis to the colouring induced by the set $N_j(x)$, we obtain sets $S'_1,\ldots,S'_r$ and $W'$ satisfying the conclusion of the lemma with $n$ replaced by $|N_j(x)|$. Setting 
  $$W = W', \qquad S_j = S'_j \cup \{x\} \qquad \text{and} \qquad S_i = S'_i \qquad \text{for each } \, i \ne j,$$ 
  we see that $W$ satisfies the minimum degree condition (by the induction hypothesis), and
  $$|W| \ge \bigg( \frac{1+\eps}{r} \bigg)^{\sum_{i = 1}^r |S'_i|}|N_j(x)| \ge \bigg( \frac{1+\eps}{r} \bigg)^{\sum_{i=1}^r |S_i|}n.$$
  Since $(S_i,W)$ is a monochromatic book in colour~$i$ for each $i \in [r]$, it follows that $S_1,\dots,S_r$ and $W$ are sets with the claimed properties.
\end{proof}

% [LEMMA] Many Erdős-Szekeres steps bound
% Source: blueprint/paper/sections/section_deducing_the_bound_on_rrk.tex line 33 (\begin{lemma}\label{lem:many:ESz:steps})
% Deprecated blueprint: deprecated/05_Deducing_the_Bound_on_the_Ramsey_Number.tex line 22-33 (\begin{lemma}[Lemma 5.3])
% Lean status: ✗ Not implemented
% Label: lem:many:ESz:steps
% Dependencies: lem:ESz:steps, def:ramsey-number-off-diagonal
% TODO: Add bound for k,r ≥ 2 and ε ∈ (0,1)
% TODO: Add R_r(k-s₁,...,k-sᵣ) ≤ e^{-ε³k/2} ((1+ε)/r)^s r^{rk}
% TODO: Add condition s₁,...,sᵣ ∈ [k] with s = s₁ + ... + sᵣ ≥ ε²k

\begin{lemma}[Many Erdős-Szekeres steps bound]
  \label{lem:many:ESz:steps} % Source paper uses same label
  \paperref{Lemma 5.3}
  \uses{lem:ESz:steps, def:ramsey-number-off-diagonal}
  %
  Let\/ $k,r \ge 2$ and\/ $\eps \in (0,1)$. We have 
  %
  $$R_r\big( k - s_1, \ldots, k - s_r \big) \le e^{-\eps^3 k / 2} \bigg( \frac{1+\eps}{r} \bigg)^s r^{rk}$$
  %
  for every $s_1,\ldots,s_r \in [k]$ with\/ $s = s_1 + \cdots + s_r \ge \eps^2 k$.
\end{lemma}
%
\begin{proof}
  By the Erd\H{o}s--Szekeres bound, we have
  $$R_r\big( k - s_1, \ldots, k - s_r \big) \le r^{rk - s}.$$
  The claimed bound follows, since $( 1 + \eps )^{\eps^2 k} \ge e^{\eps^3 k/2}$ for every $\eps \in (0,1)$.
\end{proof}

% [THEOREM] Quantitative version of main theorem
% Source: blueprint/paper/sections/section_deducing_the_bound_on_rrk.tex line 5 (\begin{theorem}\label{thm:Ramsey:multicolour:quant})
% Deprecated blueprint: deprecated/05_Deducing_the_Bound_on_the_Ramsey_Number.tex line 34-42 (\begin{theorem}[Theorem 5.1])
% Lean status: ✗ Not implemented
% Label: thm:Ramsey:multicolour:quant
% Dependencies: def:ramsey-number
% TODO: Add quantitative theorem for r ≥ 2 with δ = 2^{-160} r^{-12}
% TODO: Add bound R_r(k) ≤ e^{-δk} r^{rk} for k ≥ 2^{160} r^{16}
% TODO: Add explicit constants showing exponential improvement

\begin{theorem}[Quantitative main theorem]
  \label{thm:Ramsey:multicolour:quant} % Source paper uses same label
  \paperref{Theorem 5.1}
  \uses{def:ramsey-number, def:R, lem:ESz:steps, lem:many:ESz:steps, thm:book, lem:multibounds}
  %
  Let $r \ge 2$, and set $\delta = 2^{-160} r^{-12}$. Then
  %
  \begin{equation*}
    R_r(k) \le e^{-\delta k} r^{rk}
  \end{equation*}
  %
  for every $k \in \N$ with $k \ge 2^{160} r^{16}$. 
\end{theorem}
%
\begin{proof}
  Given $r \ge 2$, set $\delta = 2^{-160} r^{-12}$, let $k \ge 2^{160} r^{16}$ and $n \ge e^{-\delta k} r^{rk}$, and let $\chi$ be an arbitrary $r$-colouring of $E(K_n)$. Set $\eps = 2^{-50} r^{-4}$, and let $S_1,\ldots,S_r$ and $W$ be the sets given by Lemma~\ref{lem:ESz:steps}. Suppose first that $|S_1| + \cdots + |S_r| \ge \eps^2 k$, and observe that, by Lemma~\ref{lem:many:ESz:steps} and our lower bounds on $n$ and $|W|$, we have 
  $$|W| \ge \bigg( \frac{1+\eps}{r} \bigg)^{\sum_{j = 1}^r |S_j|} e^{-\delta k} r^{rk} \ge R_r\big( k - |S_1|, \ldots, k - |S_r| \big).$$ 
  Since $(S_i,W)$ is a monochromatic book in colour~$i$ for each $i \in [r]$, it follows that there exists a monochromatic copy of $K_k$ in $\chi$, as required.
  
  We may therefore assume from now on that  $|S_1| + \cdots + |S_r| \le \eps^2 k$. 
  %Apply Lemma~\ref{lem:random:partition} to the colouring restricted to the set $W$, let $X$ and $Y_1,\ldots,Y_r$ be the sets given by the lemma 
  In this case we set $X = Y_1 = \cdots = Y_r = W$, and apply Theorem~\ref{thm:book} to the colouring restricted to $W$ with 
  $$p = \frac{1}{r} - 2\eps, \qquad \mu = 2^{30} r^3, \qquad t = 2^{-40} r^{-3} k \qquad \text{and} \qquad m = R\big( k, \ldots, k, k - t \big).$$
  In order to apply Theorem~\ref{thm:book}, we need to check that $t \ge \mu^5/p$, and that
  $$|X| \ge \bigg( \frac{\mu^2}{p} \bigg)^{\mu r t} \qquad \text{and} \qquad |Y_i| \ge \bigg( \frac{e^{2^{13} r^3 / \mu^2}}{p} \bigg)^t \, m.$$
  The first two of these inequalities are both easy: the first holds because $k \ge 2^{160} r^{16}$, and for the second note that since $n \ge r^{rk/2}$ and $\sum_{j = 1}^r |S_j| \le rk/4$, we have
  $$|X| \ge \bigg( \frac{1+\eps}{r} \bigg)^{rk/4} r^{rk/2} \ge r^{rk/4} \ge \big( 2^{61} r^7 \big)^{2^{-10} r k} \ge \bigg( \frac{\mu^2}{p} \bigg)^{\mu r t},$$
  as required. The third inequality is more delicate: observe first that 
  $$R\big( k, \ldots, k, k-t \big) \le \binom{rk-t}{k,\dots,k,k-t} \le e^{-(r-1)t^2/3rk} r^{rk-t} \le e^{-t^2/6k} r^{rk-t}.$$
  and therefore, since $\delta \le 2^{-10} t^2/k^2$ and $p = 1/r - 2\eps \ge e^{-3\eps r} / r$, we have 
  $$n \ge e^{-\delta k} r^{rk} \ge r^t e^{t^2/8k} \cdot R\big( k, \ldots, k, k-t \big) \ge \frac{m}{p^t} \cdot \exp\bigg( \frac{t^2}{8k} - 3\eps r t \bigg).$$
  Moreover, by our choice of $t$ and $\mu$, we have 
  $$\frac{t}{8k} = \frac{1}{2^{43} r^3} \ge \frac{1}{2^{47}r^3} + \frac{1}{2^{48}r^3} = \frac{2^{13} r^3}{\mu^2} + 4\eps r.$$
  Finally, since $\sum_{j = 1}^r |S_j| \le \eps^2 k \le \eps t$, it follows that
  $$|Y_i| = |W| \ge \bigg( \frac{1+\eps}{r} \bigg)^{\eps t} n \ge \frac{m}{p^t} \cdot \exp\bigg( \frac{t^2}{8k} - 4\eps r t \bigg) \ge \bigg( \frac{e^{2^{13} r^3 / \mu^2}}{p} \bigg)^t \, m,$$
  %$$|Y_i| \ge \frac{|W|}{4r} \ge \bigg( \frac{1+\eps}{r} \bigg)^{\eps^2 k} \frac{n}{4r} \ge \frac{m}{p^t} \cdot \exp\bigg( \frac{t^2}{8k} - 4\eps r t \bigg) \ge \bigg( \frac{e^{2^{13} r^3 / \mu^2}}{p} \bigg)^t \, m,$$
  as claimed. Thus $\chi$ contains a monochromatic $(t,m)$-book, and hence, by our choice of $m$, $\chi$ contains a monochromatic copy of $K_k$, as required.
\end{proof}