% Deducing the Main Result section - Erdős-Szekeres steps and final assembly
% This section combines all results to prove the main theorem

\section{Deducing the Bound on the Ramsey Number}

In this section we will deduce Theorem~\ref{thm:Ramsey:multicolour}, the bound on the Ramsey numbers \(R_r(k)\), from Theorem~\ref{thm:book}.
We will in fact prove a stronger quantitative version of the theorem in the form of Theorem~\ref{thm:Ramsey:multicolour:quant}.

% ============================================================================
% ERDŐS-SZEKERES AUXILIARY LEMMAS
% ============================================================================

In order to apply Theorem~\ref{thm:book}, we need to find a suitable collection of sets \(X,Y_1,\ldots,Y_r\) in an arbitrary \(r\)-colouring of \(E(K_n)\). To do so, we simply run the Erd\H{o}s--Szekeres process until we find a subset of the vertex set in which each colour has density close to \(1/r\), and the colouring is close to regular in each colour.
%
\begin{lemma}[Single Erdős-Szekeres step]
  \label{lem:ESz:steps} % Source paper uses same label
  \paperref{Lemma 5.2}
  \uses{def:edge-colouring, def:colour-neighborhood, def:book}
  %
  Given \(n,r \in \N\) and \(\eps > 0\), and an \(r\)-colouring of \(E(K_n)\), the following holds. There exist disjoint sets of vertices \(S_1,\dots,S_r\) and \(W\) such that, for every \(i \in [r]\), 
  %
  \begin{equation*}
    |W| \ge \bigg( \frac{1+\eps}{r} \bigg)^{\sum_{j = 1}^r |S_j|} n \qquad \text{and} \qquad |N_i(w) \cap W| \ge \bigg( \frac{1}{r} - \eps \bigg) |W| - 1
  \end{equation*}
  % 
  for every \(w \in W\), and \((S_i,W)\) is a monochromatic book in colour~\(i\).
\end{lemma}
%
\begin{proof}
  We use induction on~\(n\). Note that if \(n \le r\) or \(r = 1\), then the sets \(S_1 = \cdots = S_r = \emptyset\) and \(W = V(K_n)\) have the required properties. We may therefore assume that \(n > r \ge 2\), and that there exists a vertex \(x \in V(K_n)\) and a colour \(\ell \in [r]\) such that
  %
  \begin{equation*}
    |N_\ell(x)| < \bigg( \frac{1}{r} - \eps \bigg) n - 1,
  \end{equation*}
  %
  and hence there exists a different colour \(j \ne \ell\) such that 
  %
  \begin{equation*}
    |N_j(x)| \ge \frac{1}{r-1} \bigg( n - \bigg( \frac{1}{r} - \eps \bigg) n \bigg) \ge \bigg( \frac{1 + \eps}{r} \bigg) n.
  \end{equation*}
  %
  Applying the induction hypothesis to the colouring induced by the set \(N_j(x)\), we obtain sets \(S'_1,\ldots,S'_r\) and \(W'\) satisfying the conclusion of the lemma with \(n\) replaced by \(|N_j(x)|\). Setting 
  %
  \begin{equation*}
    W = W', \qquad S_j = S'_j \cup \{x\} \qquad \text{and} \qquad S_i = S'_i \qquad \text{for each } \, i \ne j,
  \end{equation*}
  % 
  we see that \(W\) satisfies the minimum degree condition (by the induction hypothesis), and
  %
  \begin{equation*}
    |W| \ge \bigg( \frac{1+\eps}{r} \bigg)^{\sum_{i = 1}^r |S'_i|}|N_j(x)| \ge \bigg( \frac{1+\eps}{r} \bigg)^{\sum_{i=1}^r |S_i|}n.
  \end{equation*}
  %
  Since \((S_i,W)\) is a monochromatic book in colour~\(i\) for each \(i \in [r]\), it follows that \(S_1,\dots,S_r\) and \(W\) are sets with the claimed properties.
\end{proof}

The next lemma, which is an immediate consequence of the Erd\H{o}s--Szekeres bound on the \(r\)-colour Ramsey numbers, implies (roughly speaking) that either the sets \(S_1,\dots,S_r\) given by Lemma~\ref{lem:ESz:steps} satisfy \(|S_i| = o(k)\) for each \(i \in [r]\), or we are already done.
%
\begin{lemma}[Many Erdős-Szekeres steps bound]
  \label{lem:many:ESz:steps} % Source paper uses same label
  \paperref{Lemma 5.3}
  \uses{lem:ESz:steps, def:ramsey-number-off-diagonal}
  %
  Let \(k,r \ge 2\) and \(\eps \in (0,1)\). We have 
  %
  \begin{equation*}
    R_r\big( k - s_1, \ldots, k - s_r \big) \le e^{-\eps^3 k / 2} \bigg( \frac{1+\eps}{r} \bigg)^s r^{rk}
  \end{equation*}
  %
  for every \(s_1,\ldots,s_r \in [k]\) with \(s = s_1 + \cdots + s_r \ge \eps^2 k\).
\end{lemma}
%
\begin{proof}
  By the Erd\H{o}s--Szekeres bound, we have
  %
  \begin{equation*}
    R_r\big( k - s_1, \ldots, k - s_r \big) \le r^{rk - s}.
  \end{equation*}
  %
  The claimed bound follows, since \(( 1 + \eps )^{\eps^2 k} \ge e^{\eps^3 k/2}\) for every \(\eps \in (0,1)\).
\end{proof}

\begin{theorem}[Quantitative main theorem]
  \label{thm:Ramsey:multicolour:quant} % Source paper uses same label
  \paperref{Theorem 5.1}
  \uses{def:ramsey-number, lem:ESz:steps, lem:many:ESz:steps, thm:book, lem:multibounds}
  %
  Let \(r \ge 2\), and set \(\delta = 2^{-160} r^{-12}\). Then
  %
  \begin{equation*}
    R_r(k) \le e^{-\delta k} r^{rk}
  \end{equation*}
  %
  for every \(k \in \N\) with \(k \ge 2^{160} r^{16}\). 
\end{theorem}
%
\begin{proof}
  Given \(r \ge 2\), set \(\delta = 2^{-160} r^{-12}\), let \(k \ge 2^{160} r^{16}\) and \(n \ge e^{-\delta k} r^{rk}\), and let \(\chi\) be an arbitrary \(r\)-colouring of \(E(K_n)\). Set \(\eps = 2^{-50} r^{-4}\), and let \(S_1,\ldots,S_r\) and \(W\) be the sets given by Lemma~\ref{lem:ESz:steps}. Suppose first that \(|S_1| + \cdots + |S_r| \ge \eps^2 k\), and observe that, by Lemma~\ref{lem:many:ESz:steps} and our lower bounds on \(n\) and \(|W|\), we have 
  %
  \begin{equation*}
    |W| \ge \bigg( \frac{1+\eps}{r} \bigg)^{\sum_{j = 1}^r |S_j|} e^{-\delta k} r^{rk} \ge R_r\big( k - |S_1|, \ldots, k - |S_r| \big).
  \end{equation*}
  % 
  Since \((S_i,W)\) is a monochromatic book in colour~\(i\) for each \(i \in [r]\), it follows that there exists a monochromatic copy of \(K_k\) in \(\chi\), as required.
  
  We may therefore assume from now on that  \(|S_1| + \cdots + |S_r| \le \eps^2 k\). 
  %Apply Lemma~\ref{lem:random:partition} to the colouring restricted to the set \(W\), let \(X\) and \(Y_1,\ldots,Y_r\) be the sets given by the lemma 
  In this case we set \(X = Y_1 = \cdots = Y_r = W\), and apply Theorem~\ref{thm:book} to the colouring restricted to \(W\) with 
  %
  \begin{equation*}
    p = \frac{1}{r} - 2\eps, \qquad \mu = 2^{30} r^3, \qquad t = 2^{-40} r^{-3} k \qquad \text{and} \qquad m = R\big( k, \ldots, k, k - t \big).
  \end{equation*}
  %
  In order to apply Theorem~\ref{thm:book}, we need to check that \(t \ge \mu^5/p\), and that
  %
  \begin{equation*}
    |X| \ge \bigg( \frac{\mu^2}{p} \bigg)^{\mu r t} \qquad \text{and} \qquad |Y_i| \ge \bigg( \frac{e^{2^{13} r^3 / \mu^2}}{p} \bigg)^t \, m.
  \end{equation*}
  %
  The first two of these inequalities are both easy: the first holds because \(k \ge 2^{160} r^{16}\), and for the second note that since \(n \ge r^{rk/2}\) and \(\sum_{j = 1}^r |S_j| \le rk/4\), we have
  %
  \begin{equation*}
    |X| \ge \bigg( \frac{1+\eps}{r} \bigg)^{rk/4} r^{rk/2} \ge r^{rk/4} \ge \big( 2^{61} r^7 \big)^{2^{-10} r k} \ge \bigg( \frac{\mu^2}{p} \bigg)^{\mu r t},
  \end{equation*}
  %
  as required. The third inequality is more delicate: observe first that 
  %
  \begin{equation*}
    R\big( k, \ldots, k, k-t \big) \le \binom{rk-t}{k,\dots,k,k-t} \le e^{-(r-1)t^2/3rk} r^{rk-t} \le e^{-t^2/6k} r^{rk-t}.
  \end{equation*}
  %
  and therefore, since \(\delta \le 2^{-10} t^2/k^2\) and \(p = 1/r - 2\eps \ge e^{-3\eps r} / r\), we have 
  %
  \begin{equation*}
    n \ge e^{-\delta k} r^{rk} \ge r^t e^{t^2/8k} \cdot R\big( k, \ldots, k, k-t \big) \ge \frac{m}{p^t} \cdot \exp\bigg( \frac{t^2}{8k} - 3\eps r t \bigg).
  \end{equation*}
  %
  Moreover, by our choice of \(t\) and \(\mu\), we have 
  %
  \begin{equation*}
    \frac{t}{8k} = \frac{1}{2^{43} r^3} \ge \frac{1}{2^{47}r^3} + \frac{1}{2^{48}r^3} = \frac{2^{13} r^3}{\mu^2} + 4\eps r.
  \end{equation*}
  %
  Finally, since \(\sum_{j = 1}^r |S_j| \le \eps^2 k \le \eps t\), it follows that
  %
  \begin{equation*}
    |Y_i| = |W| \ge \bigg( \frac{1+\eps}{r} \bigg)^{\eps t} n \ge \frac{m}{p^t} \cdot \exp\bigg( \frac{t^2}{8k} - 4\eps r t \bigg) \ge \bigg( \frac{e^{2^{13} r^3 / \mu^2}}{p} \bigg)^t \, m,
  \end{equation*}
  %
  as claimed. Thus \(\chi\) contains a monochromatic \((t,m)\)-book, and hence, by our choice of \(m\), \(\chi\) contains a monochromatic copy of \(K_k\), as required.
\end{proof}